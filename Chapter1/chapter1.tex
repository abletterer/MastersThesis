% \pagebreak[4]
% \hspace*{1cm}
% \pagebreak[4]
% \hspace*{1cm}
% \pagebreak[4]

\chapter{Outils multidimensionnels}

\graphicspath{{Chapter1/Chapter1Figs/PNG/}{Chapter1/Chapter1Figs/PDF/}{Chapter1/Chapter1Figs/}}

Ce chapitre réalise un état de l'art des différentes déformations
spatiales, d'après \cite{GB08}, tout en les comparant, dans chaque dimension, selon des critères
de souplesse, facilité d'utilisation, efficacité (en complexité) et
exactitude. Ce classement va nous permettre de choisir le meilleur
outil pour chaque dimension, afin d'obtenir à la fin une sélection de
4 outils de dimensions différentes sur lesquels nous allons nous baser
pour réaliser les mélanges.

\section{Déformation à base de volumes}

\section{Déformation à base de surfaces}
La difficulté des déformations à base de surfaces réside dans la
manière d'attacher les points de l'espace à une surface.

\begin{itemize}
\item{\textbf{Carreau paramétrique :}} \cite{JLQ96} a été le premier à
  proposer une solution, celle d'utiliser un carreau B-spline sur
  lequel sont projetés les points de l'espace, le long de la normale
  au plan du carreau. Ainsi pour déformer l'espace, l'utilisateur n'a
  plu qu'à déformer les points de contrôle du carreau.
  \begin{itemize}
  \item{\textit{Création de l'outil de déformation :}} Les points de
    contrôle du carreau initial \( \mathcal{S}(u,v) \) sont disposés
    sur le plan XZ (Y=1). Ce carreau définit une région déformable
    délimitée par les bords du carreau (en X et en Z) et s'étendant à
    l'infini en Y.
  \item{\textit{Association des points de l'espace à l'outil :}} Pour
    chaque point de l'espace \( X = (x,y,z) \) se situant dans la
    région contenue, des coordonnées paramétriques \( U = (u,v,w) \)
    sont obtenues par calcul de la projection \( X_p \) du point \( X
    \) sur le carreau le long de la normale à ce dernier. \( (u,v) \)
    est obtenu par méthode numérique de façon à ce que \(
    \mathcal{S}(u,v) = X_p \), et \( w \) correspond à la distance de
    \( X \) à \( X_p \). On obtient donc :
    \begin{equation}
      X = \mathcal{S}(u,v) + w \cdot \mathcal{N}(u,v)
    \end{equation}
    où \( \mathcal{N}(u,v) \) est le vecteur normal au point \( (u,v)
    \) sur la surface \( \mathcal{S} \).
  \item{\textit{L'outil est modifié :}} Les points de contrôle sont
    déplacés aux positions choisies par l'utilisateur, formant un
    nouveau carreau \( \tilde{\mathcal{S}} \).
  \item{\textit{L'espace est déformé :}} Les coordonnées paramétriques
    calculées pour chaque point de l'espace sont invariantes par
    déformation, ainsi les nouvelles positions sont obtenues en
    appliquant :
    \begin{equation}
      \tilde{X} = \tilde{\mathcal{S}}(u,v) + w \cdot \tilde{\mathcal{N}}(u,v)
    \end{equation}
    où \( \tilde{\mathcal{N}}(u,v) \) est le vecteur normal au point
    \( (u,v) \) de la surface déformée \( \tilde{X} \)
  \end{itemize}
\item{\textbf{Outil étoilé :}} Un polygone de forme étoilée est un
  polygone contenant, en son intérieur, une région dite "étoilée". On
  définit une région étoilée comme étant une région depuis laquelle un
  rayon émis dans n'importe qu'elle direction n'intersecte le bord du
  polygone qu'une seule fois. Cette propriété est utile dans le
  domaine de la déformation car elle permet d'obtenir une unique
  paramétrisation en coordonnées polaires des points de l'espace
  \cite{JL00}.
  \begin{itemize}
  \item{\textit{Création de l'outil de déformation :}} L'utilisateur
    définit une surface \( S \), représentant un polyèdre étoilé. il
    définit un centre \( O \) situé dans la partie étoilée du
    polyèdre. La surface peut avoir n'importe qu'elle représentation,
    à partir du moment ou celle-ci permet de réaliser des tests
    d'intersection entre un rayon et le bord de la surface.
  \item{\textit{Association des points de l'espace à l'outil :}} Un
    rayon est construit, partant de \( O \) passant par un point de
    l'espace \( X \) et intersectant le bord de \( S \). La direction
    du rayon \( U \) et le ratio de distances de \( O \) à \( X \) et
    de \( O \) à \( P \) sont enregistrés.
  \item{\textit{L'outil est modifié :}} Une surface de destination \(
    \tilde{S} \), avec un centre \( \tilde{O} \), est définie.
  \item{\textit{L'espace est déformé :}} La déformation se fait en
    construisant un rayon partant de \( \tilde{O} \)
  \end{itemize}
\end{itemize}

\section{Déformation à base de courbes}

\section{Déformation à base de points}
% ------------------------------------------------------------------------


%%% Local Variables: 
%%% mode: latex
%%% TeX-master: "../thesis"
%%% End: 
