%!TEX root = ../thesis.tex

% \pagebreak[4]
% \hspace*{1cm}
% \pagebreak[4]
% \hspace*{1cm}
% \pagebreak[4]

\chapter{Mélange d'outils}

\graphicspath{ {Chapter3/Chapter3Figs/PNG/}
  {Chapter3/Chapter3Figs/PDF/} {Chapter3/Chapter3Figs/} }

Nous avons vu dans le premier chapitre que les outils de déformation avaient
différentes caractéristiques. Celles-ci influent sur l'aspect de la
déformation engendrée par le déplacement de points de contrôle, ainsi que sur
la facilité de déformation de l'objet. Une modification grossière de
l'apparence globale de l'objet sera plus facile à réaliser avec un outil
permettant de réaliser des déformations globale ayant une faible résolution. A
l'inverse, une modification fine un ensemble de points précis de l'objet
nécessitera l'utilisation d'un outil de déformation ayant une zone d'influence
limitée et une résolution élevée. On se demande alors comment combiner
différents outils associé à un même objet, de façon à ce que la déformation
soit visuellement lisse. Autrement dit, on recherche une méthode de
combinaison de plusieurs outils qui conserve les propriétés de continuité des
outils combinés.

\section{Etat de l'art}

On peut citer \cite{JBPS11} comme étant les premiers à proposer une méthode
permettant de mélanger des outils de déformation de différentes dimensions.
C'est sur celui-ci que nous avons commencé à travailler car les résultats nous
semblaient proches de ce que nous souhaitons réaliser. Une lecture approfondie
de l'article nous a fait comprendre que la méthode n'était pas celle que nous
souhaitions. En effet, s'ils semblent s'appuyer sur des outils ayant des
dimensions différentes en fonction des zones à déformer, la gestion interne
repose uniquement sur des déformations à base de points. L'aspect
"multidimensionnel" est donc uniquement présent comme une contrainte
supplémentaire lors du calcul des coordonnées. Par exemple pour des sommets
reliés par une arête les auteurs définissent que les coordonnées évoluent de
façon linéaire le long de cette arête. De plus, pour évaluer l'influence d'un
point de contrôle sur l'espace, la technique proposée se base sur une méthode de
diffusion (nécessitant donc une discrétisation de l'espace). Or un des critères
essentiels de notre travail est la minimisation des temps de calcul, c'est
pourquoi nous avons décidé de pas continuer à étudier cette technique et à nous
intéresser à un autre travail du domaine.

\cite{GPCP13} quant à eux, proposent une méthode permettant le mélange
d'outils de même dimension, en s'intéressant particulièrement aux cas des
déformations à base de cage. Nous nous sommes intéressés à cet article de par
sa récente publication, sa proximité avec \cite{Hur12}, un travail réalisé par
un étudiant en Master ISI en 2012, et de l'utilisation de cages de
déformation, le modèle semblant le plus utilisé ces dernières années parmi les
outils à base de surfaces. L'idée est de réaliser un pavage de différentes
cages collées ensemble le long de leurs arêtes sur tout l'espace à déformer et
de considérer la position d'un point de l'espace, et pas seulement par rapport
à sa cage \textit{propre} (à comprendre la cage englobant le point de
l'espace) mais aussi par rapport aux cages adjacentes à celle-ci. Cette
technique permet de localiser la déformation engendrée par un sommet d'une
cage sur la zone couverte par sa cage propre et l'ensemble des cages
incidentes à celle-ci. Cette méthode impose de placer des cages sur l'ensemble
de l'objet, peu importe la déformation à appliquer dessus. De plus, il faut
que les cages créées soient collées ensembles le long de leur arêtes, ce qui
peut être contraignant lors de la création de celles-ci.

Etant donné que les solutions existantes ne correspondaient pas vraiment à ce
que nous souhaitions, nous avons décidé de travailler sur une autre approche.
Cela n'a pas pour but de remplacer les méthodes existantes, mais d'apporter une
autre vision au niveau des mélanges possibles d'outils de déformation.

\section{Méthode proposée}

Nous nous sommes concentrés sur des déformations à base de cage, certaines
idées de \cite{GPCP13} nous ont semblé intéressantes, et nous avons décidé de
nous en inspirer. Notre contribution se décompose en 2 parties :

\begin{enumerate}

\item Modifier la zone d'influence des déformations à base de cages

\item Combiner les effets des déformations appliquées par les différentes cages

\end{enumerate}

\subsection{Déformation étendue}

Les déformations à base de cage sont considérées comme des déformations
globales, car la modification de la position d'un sommet de la cage affecte
tout l'espace contenu à l'intérieur de la cage. Aussi, les cages ont toujours
eu comme restriction d'englober strictement l'objet à déformer, car la
transition entre la déformation à l'intérieur et à l'extérieur de la cage
n'est pas visuellement lisse. Cela est dû au fait que la fonction résultant de
la déformation n'est pas dérivable au niveau des sommets de la cage.

Mais pourquoi les déformations à base de cage ne pourraient pas avoir le même
comportement que les déformations à base de points? A savoir que la
modification de la position d'un point de contrôle affecte l'espace autour de
sa position dans un rayon donné, en diminuant son influence progressivement en
fonction de la distance d'un point de l'espace au point de contrôle. On aurait
alors une cage qui déformerait à la fois les points de l'espace en son
intérieur et à l'extérieur. Son influence diminuerait progressivement au fur
et à mesure que les points de l'espace sont éloignés du centre de la cage,
jusqu'à n'avoir plus aucune influence au delà d'un certain rayon.

Comme expliqué juste avant, la transition de la déformation entre l'intérieur
et l'extérieur d'une cage n'est pas visuellement lisse, on ne peut donc pas
utiliser directement des coordonnées que l'on calculerait aussi bien pour les
points de l'espace à l'intérieur que ceux à l'extérieur. 

On cherche donc une fonction de déformation qui soit au moins $C^1$ dans tout
son domaine et qui puisse utiliser les méthodes de calcul de coordonnées
existantes, comme les MVC par exemple.



% \section{Cheminement de l'article}
% Dans un premier temps, nous avons voulu reproduire le cheminement de
% \cite{GPCP13}, pour comprendre quelles étaient les motivations
% derrière la définition de chaque fonction. Pour visualiser les
% différentes fonctions, nous considérons dans nos exemples un outil
% composé d'une grille de 2*2 faces, où chaque face représente une
% cage. L'unique cage jointure de chaque face est donc la cage composée
% de l'union des 4 cages initiales.

% Nous nous sommes basés sur la formulation globale décrite par
% l'article, qui exprime la position finale d'un point de l'espace comme
% étant une interpolation linéaire de la position d'un sommet $p$ par
% rapport à sa cage propre et du mélange de la position de $p$ par
% rapport à chacun de ses cages jointure:

% \begin{equation}
%   p = \beta T(p)  + (1 - \beta) J(p),
%   \label{MELgen}
% \end{equation}

% où $T(p)$ et $J(p)$ représentent la position du point p par rapport à
% sa cage propre et au mélange des positions par rapport à chaque cage
% jointure respectivement, tandis que $\beta$ représente la distance au
% bord de la cage propre.

% Le calcul de $\beta$ est proposé par l'article. Il s'agit de mesurer
% la distance d'un point de l'espace par rapport à chaque arête à la
% fois incidente à sa cage propre et à une autre cage. Pour éviter de
% devoir calculer des distances euclidiennes, l'article propose de se
% baser sur les coordonnées calculées pour chaque point $p$ par rapport
% aux sommets de sa cage propre. Comme nous l'avons expliqué au début,
% la position d'un point $p$ par rapport à une cage est calculée comme
% étant une combinaison linéaire pondérée des positions des sommets de
% cette cage. De ce fait, on peut considérer la distance de $p$ à un
% sommet de la cage comme étant liée à la coordonnée qui a été associée
% à ce sommet. Et par extension, on peut considérer la distance de $p$ à
% une arête comme étant liée à la somme des coordonnées associées aux
% sommets incidents de cette arête. Plus précisément, plus on est proche
% d'une arête, plus la somme des coordonnées associées aux sommets
% incidents à cette arête va être proche de 1. A l'inverse, plus on est
% loin d'une arête, plus la somme des coordonnées associées aux sommets
% de cette arête va être proche de 0. La distance à une arête correspond
% donc au complément à 1 de la somme des coordonnées associées à chaque
% sommet des arêtes incidentes à plusieurs cages :

% \begin{equation}
%   d_e~ = (1 - \sum_{v \in e} \lambda_v)
% \end{equation}

% Où $e$ correspond à une arête, $v$ un sommet de $e$ et $\lambda_v$ la
% coordonnée associée au sommet v. L'article calcule $\beta$ comme étant
% le produit de la distance de $p$ à chaque arête de la cage propre qui
% sont aussi incidentes à d'autres cages :

% \begin{equation}
%   \beta = f(\prod_{e \in C} d_e)
% \end{equation}

% Où $f(x)$ est une fonction de lissage (Equation \ref{MELlis}),
% permettant de faire varier la taille de la zone d'infuence du mélange,
% tout en conservant un estompement progressif :

% \begin{equation}
%   f(x) = \frac{1}{2} sin(\pi(\frac{x}{h}-\frac{1}{2})) + \frac{1}{2}
%   \label{MELlis}
% \end{equation}
% Où $h \in~ ]0,1]$ représente la zone d'influence de l'arête, ce qui
% permet de délimiter la zone où le mélange de coordonnées doit être
% fait (figure \ref{MELpar}). Sur la fonction en elle-même, l'influence
% de $h$ correspond à une contraction, en passant du domaine [0,1] au
% domaine [0,h] (Figure \ref{MELfon}).

% \begin{figure}[ht]
%   \begin{center}
%     \includegraphics[scale=0.35]{starCage-0-2}
%     \includegraphics[scale=0.35]{starCage-0-4}
%     \includegraphics[scale=0.35]{starCage-0-6}
%     \caption{Fonction de bordure calculée pour 4 cages (arêtes
%       noires). Les variations de couleur représentent les variations
%       de valeur de $\beta$ (la couleur bleu marine représentant une
%       valeur de 0 et la couleur rouge une valeur de 1). En bleu les
%       zones dites "de bordure" (i.e. points de l'espace proches d'une
%       arête incidente à une autre cage). Valeurs de h : 0.2, 0.4 et
%       0.6 pour les images à gauche, au centre et à droite
%       respectivement.}
%     \label{MELpar}
%   \end{center}
% \end{figure}

% \begin{figure}[ht]
%   \begin{center}
%     \includegraphics{chapter3-attenuationfonction}
%     \caption{Visualisation de la fonction f(x) pour différentes
%       valeurs de h}
%     \label{MELfon}
%   \end{center}
% \end{figure}

% Comme $0 \leq \beta \leq 1$, on peut considérer $\beta$ comme le
% pourcentage d'utilisation des coordonnées calculées par rapport à la
% cage propre. D'après l'équation \ref{MELgen}, lorsque $\beta$ vaut 0
% (i.e. $p$ se trouve sur une arête de la cage), la position de $p$
% dépend uniquement de la coordonnée calculée par rapport à la cage
% jointure.

% C'est là qu'apparaît le premier problème : Dans la configuration
% choisie (grille de 2*2 cages), toutes les cages sont incidentes à un
% même sommet $s$. De ce fait, la zone autour des arêtes incidentes à ce
% sommet est fortement influencée par l'unique cage jointure composée de
% l'union des 4 cages. Afin que la cage jointure reste un polygone, $s$
% ne doit pas être considéré comme un sommet de la cage jointure, car il
% fait partie de l'intérieur de celle-ci. L'équation \ref{MELgen} nous
% indique que les zones les plus proches des arêtes ne sont influencées
% que par la cage jointure. Et que, par conséquent, $s$ n'a aucune
% influence sur ces points. On peut voir sur la figure \ref{MELjoi} que
% la déformation engendrée par le déplacement de $s$ n'est pas du tout
% celle attendue.

% \begin{figure}[ht]
%   \begin{center}
%     \includegraphics[scale=0.35]{starCage-jointure}
%     \includegraphics[scale=0.35]{starCage-jointure-deformation}
%     \caption{A gauche la grille avant déformation et à droite la même
%       grille après déplacement de $s$. Les boules rouges représentent
%       les sommets de la cage jointure et la boule bleue représente le
%       sommet $s$. Les zones en bleu ne sont pas modifiées par la
%       translation de $s$.}
%     \label{MELjoi}
%   \end{center}
% \end{figure}

% De là, apparaît dans l'article la nécessité de rajouter un
% comportement spécifique pour ces fameux sommets qui font partie de
% l'intérieur de leur cage jointure. Ils résolvent ce problème au
% travers de l'expression d'une déformation à base de points, associée
% au sommet $s$. A partir de là, nous comprenons que la méthode proposée
% \cite{GPCP13} n'est que le résultat d'une suite de résolutions de cas
% spécifiques. Ce qui ne nous intéresse pas, car un de nos critères
% principaux est l'utilisation d'une formulation simple permettant
% d'exprimer les coordonnées de chaque point de façon claire. De plus,
% cette méthode semble difficilement associable avec une méthode
% multidimensionnelle, car elle se base sur une fusion d'outils (l'union
% des cages de déformation), ce qui ne semble pas être directement
% applicable pour des outils de différentes dimensions.

% Nous pouvons néanmoins relever l'avancée qu'apporte cet article dans
% le domaine des déformations à base de cages. En effet, grâce au
% mélange de plusieurs cages, les déformations peuvent être locales,
% tout en ayant un faible nombre de sommets composant les cages. De
% plus, la possibilité d'utiliser conjointement plusieurs systèmes de
% coordonnées différents permet de choisir le plus adapté aux
% différentes déformations à effectuer.

% C'est pour ça que la méthode qui suit s'inspire de cet article, en
% gardant à l'esprit cette idée de localisation de la déformation et de
% minimisation du temps de calcul des coordonnées.

% \section{Méthode proposée}
% Dans le cadre de ce stage, l'idée est de considérer des coordonnées
% \textit{étendues} pour chaque cage, au lieu de considérer des unions
% de cages, et de réaliser un mélange de coordonnées pour les points de
% l'espace qui sont sous l'influence de plusieurs cages. On se base ici
% sur le fait que les coordonnées MVC sont définies non seulement à
% l'intérieur, mais aussi à l'extérieur du polygone de contrôle.

% Pour vérifier la possibilité de réalisation d'une telle méthode, nous
% avons commencé par travailler sur un exemple simple, une cage unique
% déformant les points de l'espace à la fois à l'intérieur et à
% l'extérieur grâce aux coordonnées MVC. Il s'agissait de voir le
% comportement de la déformation, afin de savoir s'il était possible de
% calculer des coordonnées à la fois à l'intérieur et à l'extérieur de
% la cage, tout en permettant un passage lisse de l'un à l'autre. Car
% dans la littérature, les coordonnées n'ont été utilisées que pour
% déformer des points à l'intérieur du polygone et jusqu'à son bord.

% Il se trouve que les coordonnées MVC sont $C^\infty$ partout, sauf au
% niveau des sommets du polygone de contrôle, où elles ne sont que $C^0$
% (Corrolaire 4.8 \cite{HF06}). De plus, à l'extérieur de la cage, on
% peut constater que les coordonnées calculées ne diminuent pas quand la
% distance d'un de point de l'espace augmente. Ceci est dû à la
% partition de l'unité, qui est une propriété des coordonnées
% barycentriques généralisées. Cette propriété est induite par la
% normalisation des coordonnées associées à chaque sommet (Equation
% \ref{MELnor}).

% \begin{equation}
%   \lambda_i(p) = \frac{w_i(p)}{\sum_{j=0}^n w_j(p)}
%   \label{MELnor}
% \end{equation}

% Cette formule définit que la somme des coordonnées associées à chaque
% sommet de la cage, pour un point $p$ donné, doit valoir 1 (Equation
% \ref{MELsum}) :

% \begin{equation}
%   \sum_{i=0}^n \lambda_i(p) = 1
%   \label{MELsum}
% \end{equation}

% Pour expliquer en quoi cela pose un problème, regardons la
% construction des coordonnées MVC pour un point $p$ donné. Dans un
% premier temps on évalue l'influence de chaque sommet $v_i$, dont le
% calcul se fait par rapport au sommet voisins $v_{i-1}$ et $v_{i+1}$ :

% \begin{equation}
%   w_i(p) = \frac{tan(\alpha_{i-1}(p)/2) + tan(\alpha_{i}(p)/2)}{\|v_i - p\|}
% \end{equation}

% où $w_i(p)$ représente la coordonnée (avant normalisation) associée au
% sommet $v_i$ pour le point $p$ et $\alpha_i(p)$ l'angle en $p$ du
% triangle $[p,v_i,v_{i+1}]$.

% Les $w_i$ sont ensuite normalisés (Equation \ref{MELnor}), afin
% d'obtenir des coordonnées contenus dans le domaine [0,1]. Le problème
% est lié à cette normalisation, car quand on éloigne un point $p$ de la
% cage, si les $w_i(p)$ tendent vers 0 quand la distance $\|v_i - p\|$
% tend vers l'infini, leur somme aussi tend vers 0. De ce fait, les
% points extrémement éloignés de la cage seront aussi fortement
% influencés par les déformations appliquées sur la cage. On voit donc
% la nécessité de définir des zones d'influence (avec un estompement
% progressif de l'influence des coordonnées calculées) autour de chaque
% cage, afin de limiter leur champ d'action.

% Il existe aussi des problèmes avec les différentes méthodes de calcul
% des coordonnées. Premièrement, contrairement à la technique proposée
% par \cite{GPCP13}, notre méthode ne pourrait pas utiliser de
% \textit{HC} (Coordonnées Harmoniques, définies par \cite{JMDGS07}),
% car celles-ci ne sont pas définies à l'extérieur de la cage. Ce n'est
% pas un problème en soit, car cette technique nécessite une
% discrétisation de l'espace, or la minimisation du temps de calcul des coordonnées est
% une de nos contraintes. Concernant les \textit{MVC} et les \textit{GC}
% (Coordonnées de Green, définies par \cite{LLC08}), si elles sont bien
% définies dans $\mathbb{R}^2$, la fonction résultant de la déformation
% n'est pas dérivable sur tout le domaine (Figure \ref{SURcoo}). De ce
% fait, des artefacts apparaissent au niveau des points de l'espace se
% situant à proximité des sommets de la cage. Ce problème n'est pas
% encore réglé au moment de l'écriture de ce rapport, mais c'est le
% sujet des recherches et réflexions actuelles.

% \begin{figure}[ht]
%   \begin{center}
%     \begin{tabular}{|l|c|c|c|}
%       \hline
%       \textbf{Domaine} & MVC & Harmoniques & Green\\
%       \hline
%       \textbf{Intérieur} & \textcolor{OliveGreen}{$C^\infty$} 
%       & \textcolor{OliveGreen}{$C^\infty$} 
%       & \textcolor{OliveGreen}{$C^\infty$} \\
%       \hline
%       \textbf{Bord} & \textcolor{Red}{$C^0$} 
%       & \textcolor{Red}{$C^0$} 
%       & \textcolor{Red}{$C^0$} \\
%       \hline
%       \textbf{Extérieur} & \textcolor{OliveGreen}{$C^\infty$} 
%       & \textcolor{Red}{$C^0$}
%       & \textcolor{OliveGreen}{$C^\infty$} \\
%       \hline
%     \end{tabular}
%     \caption{Continuité de différentes méthodes de calcul des
%       coordonnées (d'après \cite{GPCP13})}
%     \label{SURcoo}
%   \end{center}
% \end{figure}