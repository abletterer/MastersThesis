%!TEX root = ../thesis.tex

% \pagebreak[4]
% \hspace*{1cm}
% \pagebreak[4]
% \hspace*{1cm}
% \pagebreak[4]

\chapter{Outils multidimensionnels}

\graphicspath{{Chapter1/Chapter1Figs/PNG/}{Chapter1/Chapter1Figs/PDF/}{Chapter1/Chapter1Figs/}}

Ce chapitre propose un état de l'art sur les différentes outils de déformation
spatiale, d'après le travail de \cite{GB08}. Les outils présentés
fonctionnent aussi bien dans $\mathbb{R}^2$ que dans $\mathbb{R}^3$, mais dans
ce chapitre nous considèrerons des déformations affectant l'espace
$\mathbb{R}^3$.

Tous ces outils ont différentes caractéristiques, comme par exemple la dimension
(point, courbe, surface, volume), la zone de l'espace qu'ils déforment (limitée
spatialement ou globale) ou encore leur résolution (nombre de points de contrôle
qui les composent). On dit qu'une déformation est globale lorsqu'un le
déplacement d'un sommet de l'outil influe sur l'ensemble des points de l'espace.

Ces caractéristiques nous permettent de classer les outils et d'en sélectionner
certains. Cette sélection a pour but d'établir une base sur laquelle partir lors
de la mise en place d'une méthode de mélange des déformations engendrées par les
différents outils.

\section{Déformation à base de volumes}

Les déformations à base de volumes sont définies à partir d'une grille 3D de
points de contrôle. Ces points peuvent être considérées comme des poignées que
l'on peut déplacer et qui vont modifier la position des points de l'espace.

\begin{itemize}

\item{\textbf{Déformation de forme libre de Bézier :}} \cite{SP86} ont   eu
l'idée de lier la déformation d'un volume paramétrique à l'espace   contenu à
l'intérieur de ce volume. Mathématiquement, il s'agit   d'une extension directe
des courbes de Bézier au cas 3D. Ils ont   utilisé les polynômes de Bernstein
pour définir le volume. Ce   modèle, donnant une forme parallépipédique au
volume, permet   d'obtenir une paramétrisation des coordonnées des points de
l'espace   presque automatiquement. Cette technique déforme l'espace de façon
globale.

\item{\textbf{Déformation de forme libre avec contrôle local :}} Les   volumes
paramétriques souffrent des mêmes problèmes que leurs   homologues de dimensions
inférieures (surface et courbes   paramétriques), à savoir le manque de contrôle
local. Pour corriger   le problème, une solution naïve consiste à rajouter des
points de   contrôle, afin de réduire l'impact de chacun d'entre eux sur
l'espace à déformer. Mais cette technique augmente la résolution du   volume
paramétrique et ainsi son coût en temps de calcul. Une autre   solution est
d'utiliser des fonctions définies par   morceaux. \cite{GP89} et \cite{Com89}
ont choisi de travailler avec   des B-splines, car elles sont définies
naturellement par   morceaux. Ce choix rend la déformation limitée spatialement.

\item{\textbf{Déformation de forme libre avec outil non       parallélépipédique
:}} Les déformations de forme libre ont une   grosse limitation, celle de la
forme de l'outil. En effet, comme les   volumes de déformations sont uniquement
composés de cubes alignés   sur les axes du repère de l'espace, la forme globale
correspond à un   pavé. Par conséquent, il n'est pas possible d'obtenir un outil
qui   épouse la forme de l'objet à déformer. \cite{Coq90} a donc proposé   une
version étendue des déformations de forme libre, en laissant la   possibilité
lors de la création de l'outil de modifier la position   de ses sommets avant de
l'associer à l'outil. Le calcul des   coordonnées de l'objet par rapport à
l'outil n'est donc plus   trivial, il est devenu beaucoup plus coûteux.
\cite{BBT97} ont   proposé de considérer des volumes de Bézier tétraédriques
tandis que   \cite{MJ96} se sont plutôt intéressés aux volumes de subdivision de
topologie quelconque.

\end{itemize}

\section{Déformation à base de surfaces} 

La difficulté des déformations à base de surfaces réside dans la manière
d'attacher les points de l'espace à une surface.

\begin{itemize}

\item{\textbf{Carreau paramétrique :}} \cite{JLQ96} ont été les   premiers à
proposer une solution, celle d'utiliser un carreau   B-spline sur lequel sont
projetés les points de l'espace, le long de   la normale au plan du carreau,
dans l'espace paramétrique du   carreau. Ainsi pour déformer l'espace,
l'utilisateur déplace les   différents points de contrôle. Les points de
l'espace sont   repositionnés grâce à leurs coordonnées paramétriques
(précédemment   calculées) et ``reprojetées'' le long de la normale au carreau
déformé. De même que pour les volumes paramétriques, cette technique   déforme
l'espace de façon globale.

\item{\textbf{Surface étoilée :}} Un polyèdre de forme étoilée est un   polyèdre
contenant, en son intérieur, une région qu'on appelle le   \textit{noyau}. On
définit le noyau d'un polyèdre comme étant la   région depuis laquelle un rayon
émis dans n'importe quelle direction   n'intersecte le bord du polyèdre qu'une
seule fois (Figure   \ref{SUReto}, représenté en 2D). Cette propriété est utile
dans le   domaine de la déformation car elle permet d'obtenir une unique
paramétrisation en coordonnées polaires des points de l'espace,   comme proposé
par \cite{JL00}. Cette technique déforme aussi   l'espace de façon globale.

  \begin{figure}[ht]
    \begin{center}
      \includegraphics[scale=0.5]{chapter1-pstricks.pdf}
      \caption{En rouge le noyau d'un polygone de forme étoilée
        représenté par son bord, en noir (en 2D).}
      \label{SUReto}
    \end{center}
  \end{figure}

\item{\textbf{Maillage triangulaire :}} Une autre idée est d'utiliser   un
maillage triangulaire simple pour appliquer des déformations aux   points de
l'espace. \cite{KO03} ont été les premiers à utiliser ce   genre d'outil. Ils
définissent que les triangles du maillage   contribuent à déformer une zone
sphérique autour de chacun d'entre   eux. Les triangles définissent des
coordonnées paramétriques pour   les points de l'espace se trouvant dans leur
zone d'influence, en   fonction de leur distance au triangle. La position des
points de   l'espace se trouvant à l'intersection de plusieurs zones d'influence
résultent d'une moyenne pondérée des coordonnées calculées par   rapport à
chaque triangle. L'avantage de cette technique est de   permettre une
configuration assez générale des triangles (non   nécessairement connexes), à
partir du moment où ceux-ci ne sont pas   dégénérés (les triangles doivent être
composés de 3 sommets   distincts). La déformation engendrée par le déplacement
du sommet   d'un triangle est limitée spatialement par la zone d'influence
définie autour de ce triangle.

\item{\textbf{Cage :}} La cage est l'outil surfacique le plus utilisé ces
dernières années en terme de déformation spatiale. Son utilisation est notamment
dûe aux travaux de \cite{JSW05} et de \cite{FKR05} qui ont utilisé les
\textit{Mean Value Coordinates} définies par \cite{Flo03} et \cite{FKR05}. Un
maillage surfacique triangulaire quelconque est créé et définit une position
paramétrique des points de l'espace par rapport aux positions des points de
contrôle de la cage. Plus de précisions sur cette méthode seront données dans le
prochain chapitre.

\end{itemize}

\section{Déformation à base de courbes} 

L'étude étant encore en cours, l'analyse des outils de déformation à base de
courbes n'est pas terminée. Leur explication sera donnée au rapport final.

\section{Déformation à base de points}

\begin{itemize}

\item{\textbf{Déformation radiale simple :}} Dans cette méthode, les
déformations sont déterminées par un certain nombre de contraintes,   chacune
définie par un rayon $r_i$ (permettant de faire varier la   zone d'influence)
centré sur un point de contrainte $C_i$ associé à   un déplacement $\Delta C_i$.
On définit une paramétrisation des   points de l'espace uniquement par la
distance de chacun aux points   de contraintes. De ce fait, les déformations se
répandent de façon   uniforme dans toutes les directions. \cite{BR94} ont
développé   Scodef (Simple Constrained Deformations), le premier outil utilisant
cette technique. La déformation appliquée par le déplacement d'un   point de
contrainte est limitée à la zone sphérique autour de ce   point. Il est
important de noter que ce modèle est juste une   restriction des possibilités de
déformation proposées par la   technique suivante, en considérant le cas le plus
simple avec des   zones d'influence sphériques autour des contraintes.

\item{\textbf{DOGME :}} \cite{BB91} ont introduit une méthode basée   sur les
contraintes appelée DOGME (Deformation Of Geometric Model   Editor) afin de
remplacer l'interface non-intuitive des grilles   utilisés dans les déformations
à base de volumes par une   manipulation directe des points de l'espace. L'idée
est de permettre   à un utilisateur de déplacer un point de l'espace et de
déformer son   voisinage géométrique en conséquence de façon lisse. On peut
comparer cette déformation au fait de pincer un objet pour étirer la   zone
avoisinant la partie pincée. Pour replacer le voisinage d'un   point de
l'espace, la méthode se base sur le déplacement de points   qu'on appelle
"contraintes". Des contraintes, dont la portée de   chacune est définie par une
zone d'influence, sont associées à   chaque point de l'espace et permettent
d'approcher au mieux la   nouvelle position des points. Si un point de l'espace
est sous   l'influence de plusieurs contraintes, alors sa nouvelle position
correspond à une pondération de celles-ci. De même que pour la   déformation
radiale simple, la déformation est locale, car chaque   contrainte agit
uniquement dans sa zone d'influence.

\end{itemize}