% \pagebreak[4]
% \hspace*{1cm}
% \pagebreak[4]
% \hspace*{1cm}
% \pagebreak[4]

\chapter{Outils multidimensionnels}

\graphicspath{{Chapter1/Chapter1Figs/PNG/}{Chapter1/Chapter1Figs/PDF/}{Chapter1/Chapter1Figs/}}

Ce chapitre propose un état de l'art sur les différentes déformations
spatiales, d'après \cite{GB08}, tout en les comparant, dans chaque
dimension, selon des critères de souplesse, facilité d'utilisation,
efficacité (en complexité) et exactitude. Ce classement va nous
permettre de choisir le meilleur outil associé à chaque dimension,
afin d'obtenir à la fin une sélection de 4 outils de dimensions
différentes sur laquelle nous allons nous baser pour réaliser les
mélanges. Les outils présentés fonctionnent aussi bien dans
$\mathbb{R}^2$ que dans $\mathbb{R}^3$, mais dans ce chapitre nous
nous concentrons sur des déformations affectant l'espace
$\mathbb{R}^3$.

Tous ces outils ont différentes caractéristiques, comme par exemple la
dimension topologique de l'outil (point, courbe, surface, volume) ou
de la zone de l'espace qu'il déforme (limitée spatialement ou
globale). On dit qu'une déformation est globale lorsqu'un le
déplacement d'un sommet de l'outil influe sur l'ensemble des points de
l'espace.

\section{Déformation à base de volumes}
Les déformations à base de volumes sont définies à partir d'une grille
3D de points de contrôle. Ces points peuvent être considérées comme
des poignées que l'on peut déplacer et qui vont modifier la position
des points de l'espace.

\begin{itemize}
\item{\textbf{Déformation de forme libre de Bézier :}} \cite{SP86} ont
  eu l'idée de lier la déformation d'un volume paramétrique à l'espace
  contenu à l'intérieur de ce volume. Mathématiquement, il s'agit
  d'une extension directe des courbes de Bézier au cas 3D. Ils ont
  utilisé les polynômes de Bernstein pour définir le volume. Ce
  modèle, donnant une forme parallépipédique au volume, permet
  d'obtenir une paramétrisation des coordonnées des points de l'espace
  presque automatiquement. Cette technique déforme l'espace de façon
  globale.
\item{\textbf{Déformation de forme libre avec contrôle local :}} Les
  volumes paramétriques souffrent des mêmes problèmes que leurs
  homologues de dimensions inférieures (surface et courbes
  paramétriques), à savoir le manque de contrôle local. Pour corriger
  le problème, une solution naïve consiste à rajouter des points de
  contrôle, afin de réduire l'impact de chacun d'entre eux sur
  l'espace à déformer. Mais cette technique augmente le degré du
  volume paramétrique et ainsi son coût en temps de calcul. Une autre
  solution est d'utiliser des fonctions définies par
  morceaux. \cite{GP89} et \cite{Com89} ont choisi de travailler avec
  des B-splines, car elles sont définies naturellement par
  morceaux. Ce choix rend la déformation limitée spatialement.
\item{\textbf{Déformation de forme libre avec outil non
      parallélépipédique :}} Les déformations de forme libre ont une
  grosse limitation, celle de la forme de l'outil. En effet, comme les
  volumes de déformations sont uniquement composés de cubes alignés
  sur les axes du repère de l'espace, la forme globale correspond à un
  pavé. Par conséquent, il n'est pas possible d'obtenir un outil qui
  épouse la forme de l'objet à déformer. \cite{Coq90} a donc proposé
  une version étendue des déformations de forme libre, en laissant la
  possibilité lors de la création de l'outil de modifier la position
  de ses sommets avant de l'associer à l'outil. Le calcul des
  coordonnées de l'objet par rapport à l'outil n'est donc plus
  trivial, il est devenu beaucoup plus coûteux. \cite{BBT97} ont
  proposé de considérer des volumes de Bézier tétraédriques tandis que
  \cite{MJ96} se sont plutôt intéressés aux volumes de subdivision de
  topologie quelconque.
\end{itemize}

\section{Déformation à base de surfaces}
La difficulté des déformations à base de surfaces réside dans la
manière d'attacher les points de l'espace à une surface.

\begin{itemize}
\item{\textbf{Carreau paramétrique :}} \cite{JLQ96} ont été les
  premiers à proposer une solution, celle d'utiliser un carreau
  B-spline sur lequel sont projetés les points de l'espace, le long de
  la normale au plan du carreau, dans l'espace paramétrique du
  carreau. Ainsi pour déformer l'espace, l'utilisateur déplace les
  différents points de contrôle. Les points de l'espace sont
  repositionnés grâce à leurs coordonnées paramétriques (précédemment
  calculées) et "reprojetés" le long de la normale au carreau
  déformé. De même que pour les volumes paramétriques, cette technique
  déforme l'espace de façon globale.
\item{\textbf{Surface étoilée :}} Un polyèdre de forme étoilée est un
  polyèdre contenant, en son intérieur, une région qu'on appelle le
  \textit{noyau}. On définit le noyau d'un polyèdre comme étant la
  région depuis laquelle un rayon émis dans n'importe quelle direction
  n'intersecte le bord du polyèdre qu'une seule fois (Figure
  \ref{SUReto}, représenté en 2D). Cette propriété est utile dans le
  domaine de la déformation car elle permet d'obtenir une unique
  paramétrisation en coordonnées polaires des points de l'espace,
  comme proposé par \cite{JL00}. Cette technique déforme aussi
  l'espace de façon globale.

  \begin{figure}[h]
    \begin{center}
      \scalebox{0.5} % Change this value to rescale the drawing.
      {
        \begin{pspicture}(0,-5.227397)(8.74465,5.227397)
          \definecolor{color23b}{rgb}{0.7176470588235294,0.0,0.12549019607843137}
          \definecolor{color23}{rgb}{0.7176470588235294,0.0,0.13333333333333333}
          \definecolor{color13}{rgb}{0.00392156862745098,0.00392156862745098,0.00392156862745098}
          \psline[linewidth=0.05cm,linestyle=dashed,dash=0.16cm
          0.16cm](0.8075776,3.7196715)(3.7675776,-3.0003273)
          \psline[linewidth=0.05cm,linestyle=dashed,dash=0.16cm
          0.16cm](8.687577,0.6796716)(0.6875776,-3.2003274)
          \psline[linewidth=0.05cm,linestyle=dashed,dash=0.16cm
          0.16cm](0.8075776,3.7596717)(5.9275775,3.2396717)
          \psline[linewidth=0.05cm,linestyle=dashed,dash=0.16cm
          0.16cm](8.667578,0.6996716)(3.7475777,3.9196715)
          \psline[linewidth=0.05cm,linestyle=dashed,dash=0.16cm
          0.16cm](0.0275776,-1.1003284)(6.4075775,2.3196716)
          \psline[linewidth=0.05cm,linestyle=dashed,dash=0.16cm
          0.16cm](1.6875776,1.6796716)(1.9275776,-3.0403273)
          \psline[linewidth=0.05cm,linestyle=dashed,dash=0.16cm
          0.16cm](5.587578,5.159672)(7.0668874,0.378305)
          \psline[linewidth=0.05cm,linestyle=dashed,dash=0.16cm
          0.16cm](4.0616465,-5.0348096)(6.387474,2.6586375)
          \pspolygon[linewidth=0.1,linecolor=color23,fillstyle=solid,fillcolor=color23b](2.3875775,0.1596716)(6.2477846,2.2099302)(5.2966785,-0.9603284)(3.2970715,-1.9403284)(2.3875775,0.1596716)
          \pspolygon[linewidth=0.1,linecolor=color13](0.8160028,3.7282884)(2.6309054,3.5735292)(5.571329,5.177397)(6.528022,2.1103518)(8.69465,0.6612438)(5.304018,-0.9707618)(4.023738,-5.177397)(0.0,-4.9944997)(0.028138,-1.125521)(1.7726955,-0.196966)(1.6882817,1.6601434)(0.8160028,3.7282884)
        \end{pspicture}
      }
      \caption{En rouge le noyau d'un polygone de forme étoilée
        représenté par son bord, en noir (en 2D).}
      \label{SUReto}
    \end{center}
  \end{figure}

\item{\textbf{Maillage triangulaire :}} Une autre idée est d'utiliser
  un maillage triangulaire simple pour appliquer des déformations aux
  points de l'espace. \cite{KO03} ont été les premiers à utiliser ce
  genre d'outil. Ils définissent que les triangles du maillage
  contribuent à déformer une zone sphérique autour de chacun d'entre
  eux. Les triangles définissent des coordonnées paramétriques pour
  les points de l'espace se trouvant dans leur zone d'influence, en
  fonction de leur distance au triangle. La position des points de
  l'espace se trouvant à l'intersection de plusieurs zones d'influence
  résultent d'une moyenne pondérée des coordonnées calculées par
  rapport à chaque triangle. L'avantage de cette technique est de
  permettre une configuration assez générale des triangles (non
  nécessairement connexes), à partir du moment où ceux-ci ne sont pas
  dégénérés (les triangles doivent être composés de 3 sommets
  distincts). La déformation engendrée par le déplacement du sommet
  d'un triangle est limitée spatialement par la zone d'influence
  définie autour de ce triangle.
\item{\textbf{Cage :}} La cage est l'outil surfacique le plus utilisé
  ces dernières années en terme de déformation spatiale. Son
  utilisation est notamment dûe aux travaux de \cite{JSW05} et de
  \cite{HF06} qui ont utilisé les \textit{Mean Value Coordinates}
  définies par \cite{Flo03} et \cite{FKR05} (lors de déformations dans
  $\mathbb{R}^3$). Un maillage surfacique triangulaire quelconque est
  créé et permet de définir des coordonnées paramétriques pour les
  points de l'espace par rapport aux sommets de ce maillage. Pour
  calculer ces coordonnées, il existe plusieurs méthodes, mais toutes
  sont issues des coordonnées barycentriques généralisées. Ces
  dernières sont une extension des coordonnés barycentriques
  permettant de calculer des coordonnées non plus par rapport à un
  simplexe (tétraèdre en dimension 3), mais par rapport à n'importe
  quelle cellule (polyèdre quelconque en dimension 3). On définit la
  position d'un points $p$ de l'espace comme étant une combinaison
  linéaire des positions des sommets $v_i$ du polyèdre de contrôle.
  \begin{equation}
    p = \sum_{i=0}^n \lambda_iv_i
  \end{equation}
  Où $\lambda_i$ correspond au poids associé au sommet $v_i$ pour le
  point $p$. Cette formule nous indique que la déformation est
  globale.
\end{itemize}

\section{Déformation à base de courbes}
L'étude étant encore en cours, l'analyse des outils de déformation à
base de courbes n'est pas terminée. Leur explication sera donnée au
rapport final.

\section{Déformation à base de points}
\begin{itemize}
\item{\textbf{Déformation radiale simple :}} Dans cette méthode, les
  déformations sont déterminées par un certain nombre de contraintes,
  chacune définie par un rayon $r_i$ (permettant de faire varier la
  zone d'influence) centré sur un point de contrainte $C_i$ associé à
  un déplacement $\Delta C_i$. On définit une paramétrisation des
  points de l'espace uniquement par la distance de chacun aux points
  de contraintes. De ce fait, les déformations se répandent de façon
  uniforme dans toutes les directions. Il est important de noter que
  ce modèle est juste une restriction des possibilités de déformation
  proposées par la technique suivante, en considérant le cas le plus
  simple avec des zones d'influence sphériques autour des contraintes.
\item{\textbf{DOGME :}} \cite{BB91} ont introduit une méthode basée
  sur les contraintes appelée DOGME (Deformation Of Geometric Model
  Editor) afin de remplacer l'interface non-intuitive des grilles
  utilisés dans les déformations à base de volumes par une
  manipulation directe des points de l'espace. L'idée est de permettre
  à un utilisateur de déplacer un point de l'espace et de déformer son
  voisinage géométrique en conséquence de façon lisse. On peut
  comparer cette déformation au fait de pincer un objet pour étirer la
  zone avoisinant la partie pincée. Pour replacer le voisinage d'un
  point de l'espace, la méthode se base sur le déplacement de points
  qu'on appelle "contraintes". Des contraintes, dont la portée de
  chacune est définie par une zone d'influence, sont associées à
  chaque point de l'espace et permettent d'approcher au mieux la
  nouvelle position des points. Si un point de l'espace est sous
  l'influence de plusieurs contraintes, alors sa nouvelle position
  correspond à une pondération de celles-ci.
\end{itemize}

\begin{figure}[h]
  \begin{center}
    \begin{tabular}{|l|c|c|c|}
      \hline
      \textbf{Outil} & \textbf{Type d'outil} & \textbf{Déformation} & \textbf{Forme du bord} \\
      \hline
      \hline
      Bézier & V & Globale & Cubique\\
      \hline
      Contrôle local & V & Locale & Cubique\\
      \hline
      Non-parallélépipédique & V & Locale & Libre\\
      \hline
      \hline
      Carreau paramétrique & S & Globale & Rectangulaire\\
      \hline
      Surface étoilée & S & Globale & /\\
      \hline
      Maillage triangulaire & S & Locale & Union de sphères\\
      \hline
      Cage & S & Globale & Libre\\
      \hline
      \hline
      Radiale simple & P/C & Locale & Polyèdre convexe\\
      \hline
      DOGME & P/C & Locale & Cubique\\
      \hline
    \end{tabular}
    \caption{Comparaison des différents outils existants (d'après
      \cite{GB08})}
  \end{center}
\end{figure}

% ------------------------------------------------------------------------

%%% Local Variables: 
%%% mode: latex
%%% LaTeX-command: "latex -shell-escape"
%%% TeX-master: "../thesis"
%%% End: 
