% \pagebreak[4]
% \hspace*{1cm}
% \pagebreak[4]
% \hspace*{1cm}
% \pagebreak[4]

\chapter{Outils multidimensionnels}

\graphicspath{{Chapter1/Chapter1Figs/PNG/}{Chapter1/Chapter1Figs/PDF/}{Chapter1/Chapter1Figs/}}

Ce chapitre propose un état de l'art sur les différentes déformations
spatiales, d'après \cite{GB08}, tout en les comparant, dans chaque
dimension, selon des critères de souplesse, facilité d'utilisation,
efficacité (en complexité) et exactitude. Ce classement va nous
permettre de choisir le meilleur outil associé à chaque dimension,
afin d'obtenir à la fin une sélection de 4 outils de dimensions
différentes sur laquelle nous allons nous baser pour réaliser les
mélanges. L'ensemble des outils que nous considérons dans ce chapitre
déforment l'espace $\mathbb{R}^3$.

\section{Déformation à base de volumes}
Les déformations à base de volumes sont définies à partir d'un
treillis de points de contrôle. Ces points peuvent être considérées
comme des poignées que l'on peut déplacer et qui vont modifier la
position des points de l'espace.

\begin{itemize}
\item{\textbf{Déformation de forme libre de Bézier :}} \cite{SP86} ont
  eu l'idée de lier la déformation d'un volume paramétrique (qu'on
  appelle \textit{outil}) à l'espace contenu à l'intérieur de ce
  volume. Mathématiquement, il s'agit d'une extension directe des
  courbes de Bézier au cas 3D. Ils ont utilisé les polynômes de
  Bernstein pour définir le volume. Les points $X$ de l'espace sont
  donc définis comme une somme de points $P$ de contrôle, pondérés par
  des fonctions de base polynomiale, $\mathcal{N}_r^s$, d'index $r$ et
  de degré $s$ :
  \begin{equation}
    \mathcal{H}(U) = \mathcal{H}(u,v,w) = 
    \sum_{i=0}^l \sum_{j=0}^m \sum_{k=0}^n
    \mathcal{N}_i^l(u) \cdot \mathcal{N}_j^m(v) \cdot \mathcal{N}_k^n(w) \cdot P_{i,j,k} = X
  \end{equation}
  où $u,v,w \in [0,1]$ sont les coordonnées paramétriques d'un point
  $X$ à l'intérieur du volume.
\item{\textbf{Déformation de forme libre avec contrôle local :}} Les
  volumes paramétriques souffrent des mêmes problèmes que leurs
  homologues de dimensions inférieures, à savoir le manque de contrôle
  local. Pour corriger le problème, une solution naïve consiste à
  rajouter des points de contrôle, afin de limiter l'impact de chacun
  d'entre eux sur l'espace à déformer. Mais cette technique augmente
  le degré du volume paramétrique et ainsi son coût en temps de
  calcul. Une autre solution est d'utiliser des fonctions définies par
  morceaux. \cite{GP89} et \cite{Com89} ont choisi de travailler avec
  des B-splines, car elles sont définies par morceaux naturellement.
\item{\textbf{Déformation de forme libre avec outil non cubique :}}
  Les déformations de forme libre ont une grosse limitation, celle de
  la forme de l'outil. En effet, comme les volumes de déformations
  sont paramétrisés de façon linéaire et qu'ils sont composés de
  cubes, la forme globale correspond à un pavé. Par conséquent, il
  n'est pas possible de donner la même forme à l'outil que celle de
  l'objet à déformer sans diminuer la taille de chaque cube et en
  augmentant leur nombre. \cite{Coq90} a donc proposé une version
  étendue des déformations de forme libre, en laissant la possibilité
  lors de la création de l'outil de modifier la position de ses
  sommets avant de l'associer à l'outil. Ce qui a engendré des soucis
  au niveau du calcul des coordonnées des points de l'espace, qui ne
  résultaient plus d'une paramétrisation linéaire. Le calcul des
  coordonnées est donc devenu beaucoup plus coûteux. \cite{BBT97} ont
  donc proposé de considérer des volumes de Bézier tétraédriques
  tandis que \cite{MJ96} se sont plutôt intéressés aux volumes de
  subdivision de topologie quelconque.
\end{itemize}

\section{Déformation à base de surfaces}
La difficulté des déformations à base de surfaces réside dans la
manière d'attacher les points de l'espace à une surface.

\begin{itemize}
\item{\textbf{Carreau paramétrique :}} \cite{JLQ96} ont été les
  premiers à proposer une solution, celle d'utiliser un carreau
  B-spline sur lequel sont projetés les points de l'espace, le long de
  la normale au plan du carreau, dans l'espace paramétrique du
  carreau. Ainsi pour déformer l'espace, l'utilisateur déplace les
  différents points de contrôle. Les points de l'espace sont
  repositionnés grâce à leurs coordonnées paramétriques (précédemment
  calculées) et "reprojetés" le long de la normale au carreau déformé.
\item{\textbf{Surface étoilée :}} Un polyèdre de forme étoilée est un
  polyèdre contenant, en son intérieur, une région dite "étoilée". On
  définit une région étoilée d'un polyèdre comme étant une région
  depuis laquelle un rayon émis dans n'importe qu'elle direction
  n'intersecte le bord du polyèdre qu'une seule fois (Figure
  \ref{SUReto}). Cette propriété est utile dans le domaine de la
  déformation car elle permet d'obtenir une unique paramétrisation en
  coordonnées polaires des points de l'espace, comme proposé par
  \cite{JL00}.

  \begin{figure}[h]
    \begin{center}
      \scalebox{0.5} % Change this value to rescale the drawing.
      {
        \begin{pspicture}(0,-5.227397)(8.74465,5.227397)
          \definecolor{color23b}{rgb}{0.7176470588235294,0.0,0.12549019607843137}
          \definecolor{color23}{rgb}{0.7176470588235294,0.0,0.13333333333333333}
          \definecolor{color13}{rgb}{0.00392156862745098,0.00392156862745098,0.00392156862745098}
          \psline[linewidth=0.05cm,linestyle=dashed,dash=0.16cm
          0.16cm](0.8075776,3.7196715)(3.7675776,-3.0003273)
          \psline[linewidth=0.05cm,linestyle=dashed,dash=0.16cm
          0.16cm](8.687577,0.6796716)(0.6875776,-3.2003274)
          \psline[linewidth=0.05cm,linestyle=dashed,dash=0.16cm
          0.16cm](0.8075776,3.7596717)(5.9275775,3.2396717)
          \psline[linewidth=0.05cm,linestyle=dashed,dash=0.16cm
          0.16cm](8.667578,0.6996716)(3.7475777,3.9196715)
          \psline[linewidth=0.05cm,linestyle=dashed,dash=0.16cm
          0.16cm](0.0275776,-1.1003284)(6.4075775,2.3196716)
          \psline[linewidth=0.05cm,linestyle=dashed,dash=0.16cm
          0.16cm](1.6875776,1.6796716)(1.9275776,-3.0403273)
          \psline[linewidth=0.05cm,linestyle=dashed,dash=0.16cm
          0.16cm](5.587578,5.159672)(7.0668874,0.378305)
          \psline[linewidth=0.05cm,linestyle=dashed,dash=0.16cm
          0.16cm](4.0616465,-5.0348096)(6.387474,2.6586375)
          \pspolygon[linewidth=0.1,linecolor=color23,fillstyle=solid,fillcolor=color23b](2.3875775,0.1596716)(6.2477846,2.2099302)(5.2966785,-0.9603284)(3.2970715,-1.9403284)(2.3875775,0.1596716)
          \pspolygon[linewidth=0.1,linecolor=color13](0.8160028,3.7282884)(2.6309054,3.5735292)(5.571329,5.177397)(6.528022,2.1103518)(8.69465,0.6612438)(5.304018,-0.9707618)(4.023738,-5.177397)(0.0,-4.9944997)(0.028138,-1.125521)(1.7726955,-0.196966)(1.6882817,1.6601434)(0.8160028,3.7282884)
        \end{pspicture}
      }
      \caption{En rouge la région étoilée d'un polygone représenté par
        son bord, en noir (en 2D).}
      \label{SUReto}
    \end{center}
  \end{figure}

\item{\textbf{Maillage triangulaire :}} Une autre idée est d'utiliser
  un maillage triangulaire simple pour appliquer des déformations aux
  points de l'espace. \cite{KO03} ont été les premiers à utiliser ce
  genre d'outil. Ils définissent que les triangles du maillage
  contribuent à déformer une zone de l'espace proche en réalisant une
  moyenne pondérée, et que chacun d'entre eux influent sur une zone
  sphérique autour d'eux. L'avantage de cette technique est de
  permettre une configuration assez générale des triangles (non
  nécessairement connexes), à partir du moment où ceux-ci ne sont pas
  dégénérés (les triangles doivent être composés de 3 sommets
  distincts).
\item{\textbf{Cage :}} La cage est l'outil surfacique le plus utilisé
  ces dernières années en terme de déformation spatiale. Son
  utilisation est notamment dûe aux travaux de \cite{JSW05} et de
  \cite{HF06} qui ont utilisé les \textit{Mean Value Coordinates}
  définies par \cite{Flo03} et \cite{FKR05} (lors de déformations dans
  $\mathbb{R}^3$). Un polyèdre de contrôle (quelconque) est créé et
  permet de définir des coordonnées paramétriques pour les points de
  l'espace par rapport aux sommets de ce polyèdre. Pour calculer ces
  coordonnées, il existe plusieurs méthodes, mais toutes sont tirées
  des coordonnées barycentriques généralisées (Figure
  \ref{SURcoo}). Ces dernières sont une extension des coordonnés
  barycentriques permettant de calculer des coordonnées non plus par
  rapport à un simplexe (tétraèdre en dimension 3), mais par rapport à
  n'importe quelle cellule (polyèdre quelconque en dimension 3). On
  définit la position d'un points $p$ de l'espace comme étant une
  combinaison linéaire des positions des sommets $v_i$ du polyèdre de
  contrôle.
  \begin{equation}
    p = \sum_{i=0}^n \lambda_iv_i
  \end{equation}
  Où $\lambda_i$ correspond au poids associé au sommet $v_i$ pour le
  point $p$. Le calcul de ce poids est donné par \cite{FKR05}.

  \begin{figure}[h]
    \begin{center}
      \begin{tabular}{|l|c|c|c|}
        \hline
        Coordonnées & MVC & Green & Harmoniques \\
        \hline
        Références & & & \\
        \hline
        Domaine & & & \\
        \hline
        Continuité & & & \\
        \hline
      \end{tabular}
      \caption{Comparaison des principales méthodes de calcul des
        coordonnées}
      \label{SURcoo}
    \end{center}
  \end{figure}

\end{itemize}

\section{Déformation à base de courbes}
\begin{itemize}
\item{\textbf{Déformation de De Casteljau généralisée :}} \cite{CR94}
  ont introduit une première version déformation à base de courbes
  qu'on peut assimiler à une déformation de forme libre restreinte. On
  peut considérer l'algorithme de De Casteljau comme une déformation
  des points du segment $[0,1]$ en points d'une courbe de
  Bézier. C'est de ce constat que se sont inspirés \cite{CR94}, en
  considérant cette fois-ci le cube $[0,1]^3$. Plus de détails sur
  cette technique sont donnés par \cite{BE01}.
\end{itemize}

\section{Déformation à base de points}
\begin{itemize}
\item{\textbf{DOGME :}} \cite{BB91} ont introduit une méthode basée
  sur les contraintes appelée DOGME (Deformation Of Geometric Model
  Editor) afin de remplacer l'interface non-intuitive des treillis
  utilisés dans les déformations à base de volumes par une
  manipulation directe des points de l'espace. L'idée est de permettre
  à un utilisateur de déplacer un point de l'espace et de déformer son
  voisinage géométrique en conséquence de façon lisse. On peut
  comparer cette déformation au fait de pincer un objet pour étirer la
  zone avoisinant la partie pincée. Pour replacer le voisinage d'un
  point de l'espace, la méthode se base sur le déplacement de points
  qu'on appelle "contraintes". Des contraintes, dont la portée de
  chacune est définie par une zone d'influence, sont associées à
  chaque point de l'espace et permettent d'approcher au mieux la
  nouvelle position des points. Si un point de l'espace est sous
  l'influence de plusieurs contraintes, alors sa nouvelle position
  correspond à une pondération de celles-ci.
\item{\textbf{Déformation de forme libre par manipulation directe:}}
  Les déformations de forme libre se basent sur la modification de
  sommets appartenant à un treillis pour déformer les points de
  l'espace. \cite{HHK92}, quant à eux, proposent d'interagir avec des
  points de l'espace afin de modifier la forme du treillis en fonction
  de contraintes associées aux points modifiés.
\item{\textbf{Déformation de forme libre de Dirichlet :}} Cette
  technique proposée par \cite{MT97} s'occupe de définir
  automatiquement les zones d'influence des contraintes à partir d'un
  ensemble de contraintes, sans intervention utilisateur. Elle se
  trouve à mi-chemin entre une déformation à base de points et à base
  de volumes, car si l'utilisateur interagit avec des points, la
  structure interne est constituée d'un groupe de volumes de Bézier
  (déterminés par une tétraédrisation de Delaunay), et des coordonnées
  paramétriques sont associées aux points de l'espace, par rapport à
  ces volumes.
\item{\textbf{Déformation radiale simple :}} Dans cette méthode, les
  déformations sont déterminées par un certain nombre de contraintes,
  chacune définie par un rayon $r_i$ (permettant de faire varier la
  zone d'influence) centré sur un point de contrainte $C_i$ associé à
  un déplacement $\Delta C_i$. On définit une paramétrisation des
  points de l'espace uniquement par la distance de chacun aux points de
  contraintes. De ce fait, les déformations se répandent de façon
  uniforme dans toutes les directions. Il est important de noter que
  ce modèle est juste une restriction des possibilités de déformation
  proposées par DOGME, en considérant le cas le plus simple avec des
  zones d'influence sphériques autour des contraintes.
\end{itemize}

\begin{figure}[h]
  \begin{center}
    \begin{tabular}{|l|c|c|c|}
      \hline
      Coordonnées & MVC & Green & Harmoniques \\
      \hline
      Références & & & \\
      \hline
      Domaine & & & \\
      \hline
      Continuité & & & \\
      \hline
    \end{tabular}
    \caption{Comparaison des différents outils existants d'après \cite{GB08}}
  \end{center}
\end{figure}

% ------------------------------------------------------------------------

%%% Local Variables: 
%%% mode: latex
%%% LaTeX-command: "latex -shell-escape"
%%% TeX-master: "../thesis"
%%% End: 
