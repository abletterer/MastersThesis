% \pagebreak[4]
% \hspace*{1cm}
% \pagebreak[4]
% \hspace*{1cm}
% \pagebreak[4]

\chapter{Outils multidimensionnels}

\graphicspath{{Chapter1/Chapter1Figs/PNG/}{Chapter1/Chapter1Figs/PDF/}{Chapter1/Chapter1Figs/}}

Ce chapitre réalise un état de l'art des différentes déformations
spatiales, d'après \cite{GB08}, tout en les comparant, dans chaque
dimension, selon des critères de souplesse, facilité d'utilisation,
efficacité (en complexité) et exactitude. Ce classement va nous
permettre de choisir le meilleur outil associé à chaque dimension,
afin d'obtenir à la fin une sélection de 4 outils de dimensions
différentes sur lesquels nous allons nous baser pour réaliser les
mélanges.

\section{Déformation à base de volumes}
\begin{itemize}
\item{\textbf{Déformation de forme libre de Bézier :}} \cite{SP86} a
  eu l'idée de lier la déformation d'un volume paramétrique (qu'on
  appelle outil) à l'espace contenu à l'intérieur de ce
  volume. Mathématiquement, il s'agit d'une extension directe des
  courbes de Bézier au cas 3D. \cite{SP86} a utilisé les polynômes de
  Bernstein pour définir le volume. Les points de l'espace $X$ sont
  donc définis comme une somme de points de contrôle $P$,
  coefficientés par des fonctions de base polynomiale,
  $\mathcal{N}_r^s$, d'index $r$ et de degré $s$ :
  \begin{equation}
    \mathcal{H}(U) = \mathcal{H}(u,v,w) = 
    \sum_{i=0}^l \sum_{j=0}^m \sum_{k=0}^n
    \mathcal{N}_i^l(u) \cdot \mathcal{N}_j^m(v) \cdot \mathcal{N}_k^n(w) \cdot P_{i,j,k} = X
  \end{equation}
  Où $u,v,w \in [0,1]$ sont les coordonnées paramétriques d'un point à
  l'intérieur du volume.
\item{\textbf{Déformation de forme libre avec contrôle local :}} Les
  volumes paramétriques souffrent des mêmes problèmes que leurs
  homologues de dimensions inférieures, à savoir le manque de contrôle
  local. Pour corriger le problème, une solution naïve est de rajouter
  des points de contrôle, afin de limiter l'impact de chacun sur
  l'espace à déformer. Mais cette technique augmente le degré du
  volume paramétrique, et de ce fait son coût en temps de calcul. Une
  autre solution est d'utiliser des fonctions définies par
  morceaux. \cite{GP89} et \cite{Com89} ont choisi de travailler avec
  des B-splines, car elles sont définies par morceaux naturellement.
\item{\textbf{Déformation de forme libre avec outil non cubique :}}
  Les déformations de forme libre ont une grosse limitation, celle de
  la forme de l'outil. En effet, comme les volumes de déformations
  sont paramétrisés de façon linéaire, ils ont la forme d'un pavé. De
  ce fait, il n'est pas possible de définir zones plus denses que
  d'autres. \cite{Coq90} a donc proposé une version étendue des
  déformations de forme libre, en laissant la possibilité lors de la
  création de l'outil de modifier la position de ses sommets avant de
  l'associer à l'outil. Ce qui a engendré des soucis au niveau du
  calcul des coordonnées des points de l'espace, qui ne résultaient
  plus d'une paramétrisation linéaire. De ce fait, le calcul des
  coordonnées est devenu beaucoup plus coûteux. \cite{BBT97} ont donc
  proposé de considérer des volumes de Bézier tétraédriques et
  \cite{MJ96} se sont plutôt intéressés aux volumes de subdivision de
  topologie quelconque.
\end{itemize}

\section{Déformation à base de surfaces}
La difficulté des déformations à base de surfaces réside dans la
manière d'attacher les points de l'espace à une surface.

\begin{itemize}
\item{\textbf{Carreau paramétrique :}} \cite{JLQ96} a été le premier à
  proposer une solution, celle d'utiliser un carreau B-spline sur
  lequel sont projetés les points de l'espace, le long de la normale
  au plan du carreau, dans l'espace paramétrique du carreau. Ainsi
  pour déformer l'espace, l'utilisateur n'a plus qu'à déformer les
  points de contrôle du carreau. Les points de l'espace sont
  repositionnés grâce à leurs coordonnées paramétriques (précédemment
  calculées) et "reprojetés" le long de la normale au carreau déformé.
  % \begin{itemize}
  % \item{\textit{Création de l'outil de déformation :}} Les points de
  %   contrôle du carreau initial \( \mathcal{S}(u,v) \) sont disposés
  %   sur le plan XZ (Y=1). Ce carreau définit une région déformable
  %   délimitée par les bords du carreau (en X et en Z) et s'étendant
  %   à
  %   l'infini en Y.
  % \item{\textit{Association des points de l'espace à l'outil :}}
  %   Pour
  %   chaque point de l'espace \( X = (x,y,z) \) se situant dans la
  %   région contenue, des coordonnées paramétriques \( U = (u,v,w) \)
  %   sont obtenues par calcul de la projection \( X_p \) du point \(
  %   X
  %   \) sur le carreau le long de la normale à ce dernier. \( (u,v)
  %   \)
  %   est obtenu par méthode numérique de façon à ce que \(
  %   \mathcal{S}(u,v) = X_p \), et \( w \) correspond à la distance
  %   de
  %   \( X \) à \( X_p \). On obtient donc :
  %   \begin{equation}
  %     X = \mathcal{S}(u,v) + w \cdot \mathcal{N}(u,v)
  %   \end{equation}
  %   où \( \mathcal{N}(u,v) \) est le vecteur normal au point \(
  %   (u,v)
  %   \) sur la surface \( \mathcal{S} \).
  % \item{\textit{L'outil est modifié :}} Les points de contrôle sont
  %   déplacés aux positions choisies par l'utilisateur, formant un
  %   nouveau carreau \( \tilde{\mathcal{S}} \).
  % \item{\textit{L'espace est déformé :}} Les coordonnées
  %   paramétriques
  %   calculées pour chaque point de l'espace sont invariantes par
  %   déformation, ainsi les nouvelles positions sont obtenues en
  %   appliquant :
  %   \begin{equation}
  %     \tilde{X} = \tilde{\mathcal{S}}(u,v) + w \cdot \tilde{\mathcal{N}}(u,v)
  %   \end{equation}
  %   où \( \tilde{\mathcal{N}}(u,v) \) est le vecteur normal au point
  %   \( (u,v) \) de la surface déformée \( \tilde{X} \)
  % \end{itemize}
\item{\textbf{Surface étoilée :}} Un polyèdre de forme étoilée est un
  polyèdre contenant, en son intérieur, une région dite "étoilée". On
  définit une région étoilée d'un polygone comme étant une région
  depuis laquelle un rayon émis dans n'importe qu'elle direction
  n'intersecte le bord du polygone qu'une seule fois. Cette propriété
  est utile dans le domaine de la déformation car elle permet
  d'obtenir une unique paramétrisation en coordonnées polaires des
  points de l'espace, comme proposé par \cite{JL00}.
  % \begin{itemize}
  % \item{\textit{Création de l'outil de déformation :}} L'utilisateur
  %   définit une surface \( S \), représentant un polyèdre étoilé. il
  %   définit un centre \( O \) situé dans la partie étoilée du
  %   polyèdre. La surface peut avoir n'importe qu'elle
  %   représentation,
  %   à partir du moment ou celle-ci permet de réaliser des tests
  %   d'intersection entre un rayon et le bord de la surface.
  % \item{\textit{Association des points de l'espace à l'outil :}} Un
  %   rayon est construit, partant de \( O \) passant par un point de
  %   l'espace \( X \) et intersectant le bord de \( S \). La
  %   direction
  %   du rayon \( U \) et le ratio de distances de \( O \) à \( X \)
  %   et
  %   de \( O \) à \( P \) sont enregistrés.
  % \item{\textit{L'outil est modifié :}} Une surface de destination
  %   \(
  %   \tilde{S} \), avec un centre \( \tilde{O} \), est définie.
  % \item{\textit{L'espace est déformé :}} La déformation se fait en
  %   construisant un rayon partant de \( \tilde{O} \) avec la même
  %   direction \( U \), on peut ainsi obtenir l'intersection \(
  %   \tilde{P} \) entre le rayon et le bord de \( \tilde{S} \). Le
  %   point déformé \( \tilde{X} \) est placé à la même distance
  %   relative entre \( \tilde{O} \) et \( \tilde{P} \).
  % \end{itemize}
\item{\textbf{Maillage triangulaire :}} Une autre idée est d'utiliser
  un maillage triangulaire simple pour appliquer des déformations aux
  points de l'espace. \cite{KO03} a été le premier à utiliser ce genre
  d'outil. Il définit que les triangles du maillage contribuent à
  déformer une zone de l'espace proche en réalisant une moyenne
  pondérée, et que chacun d'entre eux influent sur une zone sphérique
  autour d'eux. L'avantage de cette technique est de permettre une
  configuration assez générale des triangles (non nécessairement
  connexes), à partir du moment où ceux-ci ne sont pas dégénérés
  (sommets distincts).
  % \begin{itemize}
  % \item{\textit{Création de l'outil de déformation :}} L'utilisateur
  %   place un certain nombre de triangles, ils n'ont pas besoin
  %   d'être
  %   connexes et peuvent s'intersecter. La seule restriction est
  %   qu'ils
  %   ne soient pas dégénérés.
  % \item{\textit{Association des points de l'espace à l'outil :}} Un
  %   système de coordonnées est associé à chaque triangle, un sommet
  %   est choisi pour être l'origine \( O_i \), les deux arêtes
  %   incidentes à ce sommet sont les axes \( U_i \) et \( V_i \) et
  %   la
  %   normale du triangle est le dernier axe, \( W_i \). On peut noter
  %   que le système de coordonnées n'est pas orthogonal, sauf dans le
  %   cas d'un triangle rectangle. On associe à chaque point de
  %   l'espace
  %   des coordonnées paramétriques \( U = (u_i,v_i,w_i) \) locales à
  %   chaque triangle ainsi qu'un poids \( k_i \), qui décroit de
  %   façon
  %   linéaire dans une sphère d'influence autour de chaque triangle,
  %   dont la taille varie en fonction de la taille du triangle.
  % \item{\textit{L'outil est modifié :}} Les triangles sont modifiés,
  %   soit en déplaçant des sommets des triangles, soit en déplaçant
  %   des
  %   points de l'espace et en recalculant la position des triangles
  %   (par reverse-engineering) pour enfin déplacer le reste des
  %   points
  %   de l'espace.
  % \item{\textit{L'espace est déformé :}} Encore une fois, les
  %   coordonnées paramétriques calculées sont invariantes par
  %   déformation, on peut donc calculer la nouvelle position des
  %   points
  %   de l'espace par rapport à chaque triangle :
  %   \begin{equation}
  %     \tilde{X}_i = \tilde{O}_i + u_i\tilde{U}_i + v_i\tilde{V}_i +
  %     w_i\tilde{W}_i
  %   \end{equation}
  %   Au final, la position d'un point de l'espace est calculée comme
  %   une combinaison linéaire des coordonnées associées à chaque
  %   triangle :
  %   \begin{equation}
  %     \tilde{X} = \sum_i \frac{k_i\tilde{X}_i}{\sum_i k_i}
  %   \end{equation}
  % \end{itemize}
\item{\textbf{Cage :}} La cage est outil le plus utilisé ces dernières
  années en terme de déformation spatiale. Un polygone de contrôle
  (quelconque) est créé et permet de définir des coordonnées
  paramétriques pour les points de l'espace par rapport aux sommets du
  polygone. Pour calculer ces coordonnées, il existe plusieurs
  méthodes, mais dérivent toutes des coordonnées barycentriques
  généralisées. Les coordonnées barycentriques généralisées sont une
  extension des coordonnés barycentriques afin de permettre de
  calculer des coordonnées non plus par rapport à un simplexe
  (triangle en dimension 2), mais par rapport à n'importe quelle
  cellule (polygone quelconque en dimension 2).

  On définit les positions des points de l'espace $p$ comme étant une
  combinaison linéaire des positions des sommets du polygone de
  contrôle $v_i$.
  \begin{equation}
    p = \sum_{i=0}^n w_iv_i
  \end{equation}
  Où $w_i$ correspond au poids associé au sommet $v_i$ pour le point
  $p$.
\end{itemize}

\section{Déformation à base de courbes}
\begin{itemize}
\item{\textbf{Déformation par De Casteljau généralisée :}} \cite{CR94}
  a introduit une première version déformation à base de courbes qu'on
  peut assimiler à une déformation de forme libre restreinte. Mais
  plus de détails sur cette technique sont donnés par \cite{BE01}.  On
  peut considérer l'algorithme de De Casteljau comme une déformation
  des points du segment $[0,1]$ en des points d'une courbe de
  Bézier. C'est de ce constat que s'est inspiré \cite{CR94}, en
  considérant cette fois-ci le cube $[0,1]^3$.
\end{itemize}

\section{Déformation à base de points}
\begin{itemize}
\item{\textbf{DOGME :}} \cite{BB91} a introduit une méthode basée sur
  les contraintes appelée DOGME (Deformation of Geometric Model
  Editor) afin de remplacer l'interface non-intuitive des treillis
  utilisés dans les FFD par une manipulation directe des points de
  l'espace. L'idée est de laisser un utilisateur déplacer un point de
  l'espace et replacer le voisinage de ce point de façon lisse. On
  peut comparer cette déformation au fait de pincer un objet et
  d'étirer cette partie. Pour replacer le voisinage d'un point de
  l'espace, la méthode se base sur le déplacement de points qu'on
  appelle "contraintes". Des contraintes, dont la portée de chacune
  est définie par une zone d'influence, sont associées à chaque point
  de l'espace et permettent d'approximer au mieux la nouvelle position
  des points. Si un point de l'espace est sous l'influence de
  plusieurs contraintes, alors sa nouvelle position correspond à une
  pondération de ces contraintes.
\item{\textbf{Déformation de forme libre par manipulation directe:}}
  Les déformations de forme libre se basent sur la modification de
  sommets appartenant à un treillis pour déformer les points de
  l'espace. \cite{HHK92} quant à lui propose d'interagir avec des
  points de l'espace afin de modifier la forme du treillis en fonction
  de contraintes associées aux points modifiés.
\item{\textbf{Déformation de forme libre de Dirichlet :}} Cette
  technique proposée par \cite{MT97} s'occupe de définir
  automatiquement les zones d'influence des contraintes à partir d'un
  ensemble de contraintes, sans intervention utilisateur. Cette
  technique est à mi-chemin entre une déformation à base de points et
  à base de volumes, car si l'utilisateur interagit avec des points,
  la structure interne est constituée d'un groupe de volumes de Bézier
  (déterminés par une triangulation de Delaunay), et des coordonnées
  paramétriques sont associées aux points de l'espace, par rapport à
  ces volumes.
\item{\textbf{Déformation radiale simple :}} Dans cette méthode, les
  déformations sont déterminées par un certain nombre de contraintes,
  chacune définie par un rayon d'influe $r_i$ centré sur un point de
  contrainte $C_i$ associé à un déplacement $\Delta C_i$. On définit
  une paramétrisation des points de l'espace uniquement par leur
  distance aux points de contraintes. De ce fait, les déformations se
  répandent de façon uniforme dans toutes les directions.
\end{itemize}

% ------------------------------------------------------------------------

%%% Local Variables: 
%%% mode: latex
%%% TeX-master: "../thesis"
%%% End: 
