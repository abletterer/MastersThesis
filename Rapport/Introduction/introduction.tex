%%% Thesis Introduction --------------------------------------------------
\chapter{Introduction}

\graphicspath{ {Introduction/IntroductionFigs/PNG/}
  {Introduction/IntroductionFigs/PDF/}
  {Introduction/IntroductionFigs/} }

La modélisation géométrique a permis, dans un premier temps, de
représenter des modèles virtuels. Mais le besoin d'éditer ces modèles
a amené la modélisation géométrique a évoluer et à instaurer des
outils permettant de modifier ces modèles. Ainsi sont nés les outils
de \textit{déformation}. Dans le cadre de notre travail, nous ne nous
sommes uniquement intéressés aux déformations dites
\textit{spatiales}.
\\

La déformation spatiale consiste à déformer un objet en modifiant son
espace ambiant.  On notera \cite{Bar84} et \cite{SP86} comme étant les
premiers à avoir introduit ce type de déformation. Ce procédé a un
avantage considérable, la modification de l'espace. En effet comme la
déformation est réalisée sur chaque point de l'espace de façon
indépendante, elle n'est pas liée à la représentation interne de
l'objet. Cette propriété est essentielle, car elle permet d'assurer
que, peu importe la topologie existante entre les points de l'espace à
déformer, une même déformation de l'outil déformera l'espace de la
même manière.
\\

La première partie de ce travail consiste en la réalisation d'un
mélange de multiples outils de déformation, en s'inspirant des travaux
de \cite{JBPS11} et \cite{GPCP13}. En parallèle, une étude est faite
sur les outils de déformation de différentes dimensions, basée sur
\cite{GB08}, pour déterminer le meilleur outil associé à chaque
dimension (point, courbe, surface, volume). La dernière partie se
concentre plus particulièrement sur les modèles virtuels. Il s'agit de
fournir une génération automatique d'un outil multidimensionnel de
déformation associé à un modèle, en segmentant ce dernier et en
associant à chacune de ses segmentations l'outil de déformation le
plus adapté à sa forme.
\\

Beaucoup de méthodes existent pour déformer des points de l'espace,
mais ont des caractéristiques différentes les unes des autres. Que ce
soit au niveau de la complexité au temps d'association des points de
l'espace à l'outil, de la forme de la déformation engendrée par la
modification de l'outil, ou encore de l'élégance et de la simplicité
du modèle mathématique sous-jacent. Il n'existe aujourd'hui aucune
technique permettant de combiner les forces de chacune de ces méthodes
et de pouvoir en changer de façon interactive.
\\

Tout au long de ce travail, nos choix ont été motivés par la volonté
de fournir un outil permettant de déformer des points de l'espace de
manière interactive et fluide, et d'obtenir des formulations
mathématiques simples et claires. Le but de ce travail est de pouvoir
permettre à un utilisateur d'obtenir un outil de déformation
s'adaptant à ses besoins, en générant de façon automatique un outil
multidimensionnel de déformation, tout en lui laissant la possibilité
de paramétrer le comportement des différents outils utilisés.
%%% ----------------------------------------------------------------------


%%% Local Variables: 
%%% mode: latex
%%% TeX-master: "../thesis"
%%% End: 
