%!TEX root = ../thesis.tex

% \pagebreak[4]
% \hspace*{1cm}
% \pagebreak[4]
% \hspace*{1cm}
% \pagebreak[4]

\chapter{Perspectives}

\graphicspath{ {Chapter4/Chapter4Figs/PNG/}
  {Chapter4/Chapter4Figs/PDF/} {Chapter4/Chapter4Figs/} }

Dans le dernier chapitre, nous avons vu une nouvelle méthode permettant
combiner des outils de déformation à base de cage. Cette méthode utilise un
nouveau type d'outil, la double-cage. Cet outil est composé de deux cages.
L'une dite d'influence, délimitant la zone de l'espace sous l'influence de la
déformation. L'autre dite de contrôle, permettant à l'utilisateur de contrôler
la déformation à appliquer.

Pour l'instant nous n'avons travaillé qu'avec des outils de dimension 2, mais
l'objectif de ce sujet était de mettre en place un outil multidimensionnel de
déformation. Il est naturel de se demander comment notre méthode pourrait être
étendue à des outils de dimensions différentes. Deux approches sont possibles
pour généraliser ce travail :

\begin{enumerate}
\item Modifier la nature de la cage de contrôle
\item Mélanger des outils existants avec cette méthode
\end{enumerate}

Ce chapitre a pour but d'expliquer des extensions directes au travail qui a
été effectué. Ces extensions auraient pu être directement ajoutées au travail
effectué, mais par manque de temps nous les expliquons dans une section à
part.

\section{Modification de la nature de la cage de contrôle}

Nous avions choisi de travailler avec une cage comme outil de contrôle puisque
nous voulions une méthode de mélange d'outils de déformation à base de cage.
Mais rien ne nous empêche de changer la nature de cette cage de contrôle. Pour
généraliser un peu plus ces propos, nommons la cage de contrôle \textit{outil
de contrôle}. A partir du moment où un outil (point, courbe, surface) nous permet
de définir une cage d'influence (nécessaire aux méthodes d'atténuation et de
mélange), nous pouvons l'utiliser comme outil de contrôle.

Clarifions un peu ces explications. Imaginons que notre outil de contrôle soit
un point. Pour obtenir la cage d'influence associée à ce point, nous pouvons
construire un polygone régulier (de résolution quelconque) de façon à ce que
l'isobarycentre de ce polygone soit l'outil de contrôle (Figure \ref{EXTPoi}).
Le lien entre l'outil de contrôle et la cage d'influence s'établit sur le
principe que l'outil de contrôle doit rester l'isobarycentre de la cage
d'influence. La modification de la position de l'outil de contrôle implique
une modification de la position des sommets de la cage d'influence (par
invariance de l'association).

\begin{figure}[ht]
\begin{center}
  \includegraphics[scale=0.8]{chapter4-outilPoint4-pstricks}
  \includegraphics[scale=0.8]{chapter4-outilPoint6-pstricks}
  \includegraphics[scale=0.8]{chapter4-outilPoint8-pstricks}

  \caption[Cages d'influence à partir d'un point] {Différentes cages
d'influence créées à partir d'un point comme outil de contrôle.}
  \label{EXTPoi}

\end{center}
\end{figure}

Dans le même esprit, nous pourrions obtenir une cage d'influence à partir
d'une courbe. La cage d'influence serait générée comme un épaississement de la
courbe. Le lien entre la courbe et la cage d'influence pourrait se faire comme
pour les cages de contrôle. Mis à part qu'ici au lieu de modifier la position
d'un seul sommet, un point de contrôle modifierait la position des deux
sommets qui résultent de l'épaissement de la courbe au niveau de ce point de
contrôle (Figure \ref{EXTCou}).

\begin{figure}[ht]
\begin{center}
  \includegraphics[scale=0.8]{chapter4-outilCourbe4-pstricks}
  \includegraphics[scale=0.8]{chapter4-outilCourbe6-pstricks}
  \includegraphics[scale=0.8]{chapter4-outilCourbe10-pstricks}

  \caption[Cages d'influence à partir d'une courbe] {Différentes cages
d'influence créées à partir de courbes de différentes résolutions comme outils
de contrôle.}

  \label{EXTCou}

\end{center}
\end{figure}

\section{Mélange d'outils existants}

Plusieurs outils existent déjà, comme nous avons pu le voir dans le chapitre
qui fait l'état de l'art des outils de déformation spatiale. On se demande
alors si notre technique permet le mélange de certains de ces outils. Et si
c'est possible, comment notre technique pourrait permettre de mélanger ces
différents outils ensemble.

Un point important de la méthode de mélange proposée dans ce travail est la
notion de \textit{distance à l'outil}. En effet, c'est grâce à cette distance
que l'importance de l'atténuation peut être évaluée. A partir du moment où un
outil permet de connaître la position d'un point de l'espace par rapport à la
position des points de contrôle de cet outil et au bord du domaine
d'influence, la technique de mélange présentée au chapitre précédent devrait
fonctionner.

Il est évident que les outils de déformation dont l'influence est globale ne
peuvent pas être utilisés tels quels, car ils n'ont pas un domaine d'influence
fini. Néanmoins, comme nous l'avons fait pour les déformations à base de
cages, il devrait être possible de rendre leur influence locale. Une fois le
domaine d'influence défini, il serait possible d'évaluer le critère de
"distance à l'outil". Cette distance calculée nous permettrait dans un premier
temps de modifier partiellement un modèle grâce à un outil de déformation,
mais aussi de mélanger plusieurs outils ensembles (à condition que chacun
puisse définir cette notion de distance).

Ces réflexions sont purement théoriques et certaines suppositions pourraient
être fausses, néanmoins nous pensons que des réflexions plus poussées dans ces
directions seraient intéressantes pour les travaux futurs.
