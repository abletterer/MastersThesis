%!TEX root = ../thesis.tex

\def\baselinestretch{1}
\chapter{Bilan}
\ifpdf
\graphicspath{{Conclusions/ConclusionsFigs/PNG/}{Conclusions/ConclusionsFigs/PDF/}{Conclusions/ConclusionsFigs/}}
\else
\graphicspath{{Conclusions/ConclusionsFigs/EPS/}{Conclusions/ConclusionsFigs/}}
\fi

\section{Contribution}

Nos travaux ont abouti à deux contributions originales et à la création d'un
nouvel outil, la double-cage. D'une part il est désormais possible de déformer
uniquement une partie d'un objet grâce à une déformation à base de cage. Une
zone d'influence a été mise en place, de façon à ce que la déformation puisse
se produire dans une zone définie autour d'une cage. De cette façon il est
possible d'associer seulement une zone d'un objet avec une cage et d'obtenir
une transition qui soit visuellement lisse entre la zone déformée et la zone
qui n'est pas sous l'influence de la déformation. Cette configuration permet
de réaliser des déformations locales, réduisant la résolution de la cage au
minimum nécessaire pour appliquer la déformation souhaitée ainsi que le nombre
de points de l'espace à considérer lors de la déformation. Ce qui amène à une
diminution du coût en temps de calcul à l'étape d'association par rapport à la
méthode classique de déformation à base de cage.

D'autre part, nous avons proposé un nouvel outil de déformation spatiale,
composé de plusieurs outils de déformation à base de cage. L'avantage de cette
méthode par rapport au travail de \cite{GPCP13} est de permettre une
disposition quelconque (avec ou sans recouvrement) des différentes cages de
déformation. De nombreux cas spécifiques que \cite{GPCP13} ont rencontré sont
évités par notre méthode. Ce qui nous permet d'avoir une formulation
mathématique plus simple et plus concise. En comparaison au travail de
\cite{JBPS11} et \cite{GPCP13} notre technique à l'avantage de ne pas
nécessiter l'utilisation de fonctions harmoniques, ce qui permet d'obtenir un
coût en temps de calcul au temps d'association beaucoup plus faible que ces
méthodes. De plus, à la moindre modification d'un outil, ces techniques
nécessitent de recommencer depuis le début de l'étape d'association. Notre
méthode permet de localiser la zone de l'espace sur laquelle faire l'étape
d'association. Par exemple à l'ajout d'une nouvelle double-cage il est
seulement nécessaire de réaliser l'étape d'association pour les points de
l'espace qui sont sous l'influence de cette double-cage. A la prochaine étape
d'association, les calculs prendront en compte l'influence de la double-cage
créée.

\section{Discussions}

Les cages d'influence doivent être convexes dans le cas de l'utilisation de
coordonnées MVC. Comme ces coordonnées peuvent être négatives dans le cas
d'une cage concave, la distance au bord ne peut plus être évaluée
correctement. Néanmoins, les autres méthodes de calcul de coordonnées (HC et
GC) ne connaissent pas ces problèmes de coordonnées négatives, la technique
pourrait donc fonctionner avec des cages concaves en utilisant ces méthodes de
calcul de coordonnées.

La technique que nous utilisons pour lier une cage de contrôle à une cage
d'influence ne permet de conserver la propriété d'homothétie entre les deux
cages. Ce n'est pas un problème en soit, sachant que la relation d'homothétie
n'est utile que lors de la création du lien entre les deux cages. La cage
d'influence gagnerait simplement à être "mieux" déformée par rapport à la
modification de la cage de contrôle.

\section{Ouvertures}

A court terme, il faudrait essayer le fonctionnement des double-cages dans
$\mathbb{R}^3$. Il n'y a pas de raison que la méthode implémentée ne soit pas
directement généralisable, sachant qu'elle se base sur des calculs de
coordonnées qui sont bien définis dans $\mathbb{R}^3$. Il serait intéressant
d'utiliser notre technique avec d'autres méthodes de calcul de coordonnées
(comme expliqué précédemment). Notre travail s'est concentré sur les MVC, mais
nous avons présenté d'autres techniques, comme les HC ou les GC, qui ont des
propriétés intéressantes par rapport à la forme de la déformation. Notre
méthode de déformation devrait aussi fonctionner avec ces techniques de calcul
de coordonnées, cependant aucun test n'a été effectué pour vérifier ces
intuitions.

A plus long terme, il faudrait s'intéresser aux interactions utilisateur. Les
interactions avec l'outil de contrôle ne sont pas intuitives (l'utilisateur
modifie un outil, qui modifie un autre outil, qui lui-même modifie le modèle).
Pour les améliorer il serait intéressant de mettre en place une méthode
permettant de manipuler directement la position des points de l'espace, et de
modifier la position des points de son voisinage en conséquence. Le modèle
interne se baserait toujours sur des déformations à base de cage, mais l'outil
ne serait plus directement contrôlable, seule la position des points de
l'espace pourrait être modifiée.
