\def\baselinestretch{1}
\chapter{Continuité du stage}
\ifpdf
\graphicspath{{Conclusions/ConclusionsFigs/PNG/}{Conclusions/ConclusionsFigs/PDF/}{Conclusions/ConclusionsFigs/}}
\else
\graphicspath{{Conclusions/ConclusionsFigs/EPS/}{Conclusions/ConclusionsFigs/}}
\fi

% \def\baselinestretch{1.66}

Les prochaines semaines seront consacrées à l'étude de \cite{LS08} qui
permettrait de définir des coordonnées MVC interpolant non seulement les
positions des sommets du polygone de contrôle, mais aussi les dérivées
premières (tangentes) en ces sommets. . Quand le mélange de plusieurs
cages sera possible, nous nous consacrerons à l'extension aux outils
de différentes dimensions. De ce fait nous jugerons les outils les
mieux adaptés (pour chaque dimension) à mélanger avec les cages de
déformation.

%%% ----------------------------------------------------------------------

% ------------------------------------------------------------------------

%%% Local Variables: 
%%% mode: latex
%%% TeX-master: "../thesis"
%%% End: 
