%!TEX root = ../thesis.tex

\def\baselinestretch{1}
\chapter{Bilan}
\ifpdf
\graphicspath{{Conclusions/ConclusionsFigs/PNG/}{Conclusions/ConclusionsFigs/PDF/}{Conclusions/ConclusionsFigs/}}
\else
\graphicspath{{Conclusions/ConclusionsFigs/EPS/}{Conclusions/ConclusionsFigs/}}
\fi

\section{Contribution}

Nos travaux ont abouti à deux contributions. D'une part il est désormais
possible de déformer uniquement une partie d'un modèle grâce à une déformation
à base de cage. Une zone d'influence a été mise en place, de façon à ce que la
déformation puisse se produire dans une zone définie autour d'une cage. De
cette façon il est possible d'associer seulement une zone d'un modèle avec une
cage et d'obtenir une transition qui soit visuellement lisse entre la zone
déformée et la zone statique. Cette configuration permet de réaliser des
déformations locales, réduisant le coût en temps de calcul des étapes
d'association et de déformation.

D'autre part, nous avons proposé un nouvel outil de déformation spatiale,
composé de plusieurs outils de déformation à base de cage. La déformation.
L'avantage de cette méthode par rapport au travail de \cite{GPCP13} est de
permettre une disposition quelconque (avec ou sans recouvrement) des
différentes cages de déformation. En comparaison au travail de \cite{JBPS11}
et \cite{GPCP13}, notre technique à l'avantage de ne pas nécessiter
l'utilisation de fonctions harmoniques, ce qui permet d'obtenir un temps
d'association beaucoup plus rapide que ces méthodes.

\section{Discussions}

Les cages d'influence doivent être convexes dans le cas d'utilisation de
coordonnées MVC. Comme ces coordonnées peuvent être négatives dans le cas
d'une cage concave, la distance au bord ne peut plus être évaluée
correctement.

La technique que nous utilisons pour lier une cage de contrôle à une cage
d'influence ne permet de conserver la propriété d'homothétie entre les deux
cages. Ce n'est pas un problème en soit, sachant que la relation d'homothétie
n'est utile que lors de la création du lien entre les deux cages.

\section{Ouvertures}

A court terme, il serait intéressant d'utiliser notre technique avec d'autres
méthodes de calcul de coordonnées. Notre travail s'est concentré sur les MVC,
mais nous avons présenté d'autres techniques, comme les HC ou les GC, qui ont
des propriétés intéressantes par rapport à la forme de la déformation. Notre
méthode de déformation devrait aussi fonctionner avec ces techniques de calcul
de coordonnées, cependant aucun test n'a été effectué pour vérifier ces
intuitions.

Les interactions avec l'outil de contrôle ne sont pas forcément intuitives.
Pour les améliorer il serait intéressant de mettre en place une méthode
permettant de modifier la position d'un point de l'espace directement, et de
modifier la position des points de son voisinage en conséquence. Le modèle
interne se baserait toujours sur des déformations à base de cage, mais l'outil
ne serait plus directement contrôlable, seule la position des points de
l'espace pourrait être modifiée.

Il n'y a pour l'instant aucune gestion des problèmes tels que la conservation
des propriétés du modèle (volume, angles) ou la gestion des auto-intersections
du modèle sur lui-même.

%%% ----------------------------------------------------------------------

% ------------------------------------------------------------------------

