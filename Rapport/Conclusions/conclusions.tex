%!TEX root = ../thesis.tex

\def\baselinestretch{1}
\chapter{Bilan}
\ifpdf
\graphicspath{{Conclusions/ConclusionsFigs/PNG/}{Conclusions/ConclusionsFigs/PDF/}{Conclusions/ConclusionsFigs/}}
\else
\graphicspath{{Conclusions/ConclusionsFigs/EPS/}{Conclusions/ConclusionsFigs/}}
\fi

Nos travaux ont abouti à deux contributions. D'une part il est désormais
possible de déformer uniquement une partie d'un modèle grâce à une déformation
à base de cage. Une zone d'influence a été mise en place, de façon à ce que la
déformation puisse se produire dans une zone définie autour d'une cage. De
cette façon il est possible d'associer seulement une partie d'un modèle avec
une cage et d'obtenir une déformation qui soit visuellement lisse. Cette
configuration permet de réaliser des déformations locales, réduisant le coût
en temps de calcul des étapes d'association et de déformation, car moins de
points de l'espace sont à considérer.

D'autre part, nous avons proposé un nouvel outil de déformation, composé de
plusieurs outils de déformation à base de cage. La déformation. L'avantage de
cette méthode par rapport au travail de \cite{GPCP13} est de permettre une
disposition quelconque (avec ou sans recouvrement) des différentes cages de
déformation. En comparaison au travail de \cite{JBPS11} et \cite{GPCP13},
notre technique à l'avantage de ne pas avoir besoin de travailler avec des
fonctions harmoniques, ce qui permet d'obtenir des temps d'association
beaucoup plus courts que ces techniques.

Ces méthodes utilisent les qualités des méthodes de calcul de coordonnées
existantes. Notre apport se place donc comme un complément des méthodes
existantes
.
%%% ----------------------------------------------------------------------

% ------------------------------------------------------------------------

