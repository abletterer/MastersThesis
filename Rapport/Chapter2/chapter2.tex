%!TEX root = ../thesis.tex

% \pagebreak[4]
% \hspace*{1cm}
% \pagebreak[4]
% \hspace*{1cm}
% \pagebreak[4]

\chapter{Déformation à base de cage}

\graphicspath{ {Chapter2/Chapter2Figs/PNG/}
  {Chapter2/Chapter2Figs/PDF/} {Chapter2/Chapter2Figs/} }

Notre travail s'est porté particulièrement sur le mélange de différents outils
à base de surfaces. La technique retenue pour cette dimension est celle des
déformations à base de cage. Ce choix a été motivé par l'efficacité de cette
méthode et de son utilisation dans le travail de \cite{GPCP13} (un travail
similaire à celui que nous souhaitions réaliser).

Ces déformations peuvent se faire dans $\mathbb{R}^2$ (l'outil est un
polygone) et dans $\mathbb{R}^3$ (l'outil est une surface). Pour simplifier la
compréhension de la méthode ainsi que les différents schémas, nous nous
limitons aux déformations dans $\mathbb{R}^2$.

\section{Introduction} 

L'outil permettant de réaliser les déformations est un polygone. Ce polygone
peut-être de nature quelconque, sans auto-intersection et non dégénéré: il
doit avoir au moins 3 sommets. Dans la suite, nous appelons ce polygone
\textit{cage}. L'idée est de paramétriser la position des sommets d'un modèle
contenus à l'intérieur d'une cage en fonction de la position des sommets de
cette cage. Cette méthode permet de déformer un modèle uniquement en
manipulant les sommets de la cage, peu importe la complexité de ce modèle.
Cette déformation est dite globale, car un sommet de la cage a une influence
sur tous les points de l'espace (qui sont les sommets du modèle) contenus à
l'intérieur de la cage.

Les méthodes présentées ci-dessous permettent de réaliser les étapes
d'association d'un modèle à un outil et de déformation d'un modèle. Afin de
les illuster, nous allons déformer un même modèle (Figure \ref{DEFAva}) en
utilisant chaque méthode.

\begin{figure}[ht]
\begin{center}
\includegraphics[scale=0.25]{Deformation-Texte-Avant}

\caption[Texte avant déformation] {Texte avant déformation. La cage est
représentée par son bord, en orange. (Image de \cite{LLC08})}

\label{DEFAva}
\end{center}
\end{figure}

Les premiers travaux dans ce domaine ont été réalisés simultanément par
\cite{JSW05} \cite{FKR05}, en utilisant la technique des Mean Value
Coordinates (MVC) (développée par \cite{Flo03}). Il s'agit d'utiliser des
coordonnées barycentriques généralisées pour réaliser des déformations. Les
coordonnées barycentriques généralisées correspondent à la généralisation du
calcul de coordonnées barycentriques, en ne considérant plus des coordonnées
calculées par rapport à un simplexe (triangle dans $\mathbb{R}^2$), mais à une
cellule (polygone dans $\mathbb{R}^2$). Néanmoins cette méthode comporte un
important défaut : les coordonnées calculées peuvent être négatives (dans le
cas d'une cage concave par exemple), ce qui peut aboutir à des résultats
contre-intuitifs.

\begin{figure}[ht]
\begin{center}
\includegraphics[scale=0.25]{Deformation-Texte-MVC}

\caption[Déformation d'un texte (MVC)] {Déformation d'un texte en utilisant la
méthode des MVC. (Image de \cite{LLC08})}

\label{DEFMea}
\end{center}
\end{figure}

\cite{JMDGS07} ont déclaré que l'importance de ce problème devait amener à ne
pas utiliser les MVC pour l'animation de personnages. A la place, ils ont
proposé une approche différente, qui ne produisait pas de coordonnées
négatives : les Harmonic Coordinates (Figure \ref{DEFHar}). Cette méthode a
permis d'obtenir des déformations plus localisées, au prix d'une complexité en
temps de calcul au temps d'association beaucoup plus importante et d'une
discrétisation de l'espace assez fine.

\begin{figure}[ht]
\begin{center}
\includegraphics[scale=0.25]{Deformation-Texte-HC}

\caption[Déformation d'un texte (HC)] {Déformation d'un texte en utilisant la
méthode des HC. (Image de \cite{LLC08})}

\label{DEFHar}
\end{center}
\end{figure}

Dans la même optique, \cite{LKCL07} proposent une amélioration des MVC qui les
rend positives : les Positive Mean Value Coordinates. L'idée est de se baser
sur un test de visibilité lors du calcul des coordonnées. Si un point de
l'espace ne se trouve pas dans la partie de la cage visible depuis un sommet
donné, alors la coordonnée pour ce point vaut 0. Ils ont pu obtenir des
résultats similaires à \cite{JMDGS07} pour une complexité en temps de calcul
proche de celle des MVC. \cite{LLC08} ont constaté que les détails de la
surface n'étaient pas préservés, en particulier lors de déformations
importantes. Ce qui les a amené à développer une approche considérant plus de
données lors de l'association du modèle à la cage : les Green Coordinates (GC)
(Figure \ref{DEFGre}). Au lieu d'utiliser uniquement la position des sommets
de la cage, les Green Coordinates prennent aussi en compte les normales aux
faces de la cage.

\begin{figure}[ht]
\begin{center}
\includegraphics[scale=0.25]{Deformation-Texte-GC}

\caption[Déformation d'un texte (GC)] {Déformation d'un texte en utilisant la
méthode des GC. (Image de \cite{LLC08})}

\label{DEFGre}
\end{center}
\end{figure}

\section{Mean Value Coordinates}

Dans le cadre de notre travail, nous avons choisi de travailler avec une seule
méthode de calcul des coordonnées. Le choix que nous avons fait s'est porté
sur les MVC. C'est actuellement la méthode de calcul des coordonnées qui a la
plus faible complexité en temps de calcul. De plus, ayant déjà travaillé sur
cette méthode, nous avions déjà effectué une première implémentation de la
méthode. La réutiliser nous permet de plus concentrer notre travail sur la
partie mélange de coordonnées. Néanmoins, la méthode apportée par notre
travail ne se limite pas aux coordonnées MVC, mais peut être généralisée à
d'autres méthodes de calcul de coordonnées.

\subsection{Coordonnées barycentriques}

\cite{Mob27} a été le premier à exprimer une relation entre un triangle et un
point contenu à l'intérieur de celui-ci : les coordonnées barycentriques.

Soit $V = \{v_1, v_2, v_3\}$ un triangle quelconque, ses sommets constituent ce
qu'on appelle un repère barycentrique, et soit $p$ un point de l'espace. On dit
que $p$ est un barycentre de $V$ si et seulement si:

\begin{equation}
  p = \frac{w_1 v_1 + w_2 v_2 + w_3 v_3}{w_1+w_2+w_3},
  \label{DEFPos}
\end{equation}

où $w_i$ correspond à la coordonnée du point $p$ associée au sommet $v_i$,
pour $i \in \{1, 2, 3\}$. On peut réécrire cette formule en normalisant les
$w_i$ :

\begin{equation}
  \lambda_i = \frac{w_i}{w_1+w_2+w_3} ~, i \in \{1, 2, 3\}, 
\end{equation}

et donc $\sum_{i=1}^3 \lambda_i = 1$. L'équation \ref{DEFPos} devient :

\begin{equation}
  p = \lambda_1 v_1 + \lambda_2 v_2 + \lambda_3 v_3.
\end{equation}

La coordonnée $w_i$ s'obtient en normalisant les aires (signées) des triangles
$pv_1v_2$, $pv_2v_3$, $pv_3v_1$ par rapport à l'aire totale du triangle
$v_1v_2v_3$ :

\begin{equation}
  w_1 = \frac{A(pv_1v_2)}{A(pv_1v_2)+A(pv_2v_3)+A(pv_3v_1)}
\end{equation}

\subsection{Coordonnées barycentriques généralisées}

Soit $V = \{v_1, v_2, ..., v_n\}$ un polygone sans auto-intersection et non-
dégénéré, ses sommets constituent un repère barycentrique. Soit $p$ un point
de l'espace. Le but est d'étudier les coordonnées $\lambda_i$ tels que :

\begin{equation}
  p = \sum_{i=1}^{n} \lambda_i v_i ,
  ~ \forall i \in \mathbb{N} ,~ i \in \{1,\text{...}, n\}
\end{equation}

où $\lambda_i$ est la coordonnée (normalisée) du point $p$ associée au sommet
$v_i$ et vérifie :

\begin{equation}
  \lambda_i = \frac{w_i}{\sum_{j=1}^n w_j},
\end{equation}

de même que pour les coordonnées barycentriques, $\sum_{i=1}^n \lambda_i = 1$.
\\

Le calcul des coordonnées $w_i$ n'est plus aussi évident que pour le cas du
triangle. Contrairement aux coordonnées barycentriques, les coordonnées
barycentriques généralisées ne sont pas uniques. Plusieurs méthodes ont été
mises en place, et nous nous sommes intéressés particulièrement au travail de
\cite{Flo03}.

\subsection{Calcul des coordonnées}

Sa motivation a été de reproduire le comportement de fonctions harmoniques.
Une fonction $u$ définie dans un plan $\Omega \in \mathbb{R}^2$ est dite
harmonique si elle est $C^2$ et qu'elle satisfait l'équation de Laplace :

\begin{equation}
  \frac{\partial^2 u}{\partial x^2} + \frac{\partial^2 u}{\partial y^2} = 0.
\end{equation}

Les fonctions harmoniques n'ayant pas de forme analytique, une manière de
connaître leur valeur en un point précis est de les évaluer sur tout le
domaine. Ces traitements nécessitent beaucoup de calculs, c'est pour cela que
\cite{Flo03} a décidé de les approximer au travers de fonctions linéaires
définies par morceaux sur une triangulation donnée.

La coordonnnée $w_i$ associée au point $p$ est calculée en utilisant la
position du point $p$, du sommet $v_i$ et des deux sommets adjacents à $v_i$ :
$v_{i-1}$ et $v_{i+1}$ (Figure \ref{DEFcal}).

\begin{figure}[ht]
  \begin{center}
    \includegraphics[scale=0.9]{chapter2-MVC-pstricks}

    \caption[Méthode de calcul MVC] {Méthode de calcul de la coordonnée $w_i$
du point $p$ par rapport au sommet $v_i$}

    \label{DEFcal}   \end{center} \end{figure}

La méthode détermine la coordonnée $w_i$ de la manière suivante :

\begin{equation}
  w_i = \frac{tan(\alpha_{i-1}/2) + tan(\alpha_i/2)}{\| v_i - p \|}.
  \label{DEFcoo}
\end{equation}

Ces coordonnées sont ensuite normalisées :

\begin{equation}
  \lambda_i = \frac{w_i}{\sum_{j=1}^n w_j},
\end{equation}

afin d'obtenir la coordonnée MVC du point $p$ par rapport au sommet $v_i$. On
appelle les coordonnées MVC du point $p$ l'ensemble de ces coordonnées
$\{\lambda_0, \lambda_1, ..., \lambda_n\}$.

% Une fois ces calculs effectués, la déformation se réalise par invariance de
% l'association entre le modèle et l'outil. La modification de la position des
% sommets de la cage modifie donc la position des points de l'espace (Figure
% \ref{DEFCom}).

% \begin{figure}
% \begin{center}
% \includegraphics{casque-viking}

% \caption[Deformation à base de cage (MVC)] {Déformation d'un modèle (casque
% viking) par rapport à la modification d'un sommet de la cage associée (en
% noir). A gauche le modèle avant déformation, à droite le modèle après
% déformation.}

% \label{DEFCom}
% \end{center}
% \end{figure}

Ces formules sont correctes dans le cas où un point $p$ est strictement inclus
dans la cage : $p$ n'appartient à aucune arête de la cage. Des problèmes se
posent lorsqu'un point de l'espace se trouve le long d'une arête de la cage.
On peut détecter cette situation en regardant la valeur des angles
$\alpha_{i-1}$ et $\alpha_i$. Par exemple, quand le point $p$ se situe sur
l'arête $E = [v_i,v_{i+1}]$, l'angle $\alpha_i$ vaut $\pi$ (Figure
\ref{DEFinc}). Or :

\begin{displaymath}
  \lim\limits_{x \to \pi^-} tan(\frac{x}{2}) = +\infty
\end{displaymath}

\begin{figure}[ht]
  \begin{center}
    \includegraphics[scale=0.9]{chapter2-MVC-notWorking-pstricks}

    \caption[Cas particulier MVC] {Situation dans laquelle les
formules de calcul de coordonnées ne sont plus correctes.}

    \label{DEFinc}
  \end{center}
\end{figure}

Cela pose un problème par rapport au calcul donné par l'équation \ref{DEFcoo}.
Les coordonnées barycentriques généralisées établissent que les coordonnées
calculées varient de façon linéaire le long des arêtes de la cage et qu'elles
respectent la propriété de Lagrange aux sommets de la cage: Si $p = v_i$ alors
$\lambda_i = 1$ et $\lambda_j = 0 ~\forall~ j \neq i$

En utilisant ces propriétés, on définit un calcul de coordonnées spécifique
pour les points se trouvant le long d'une arête. Pour calculer la coordonnée
$w_j$ du point $p$ par rapport au sommet $v_j$, $p$ se trouvant le long de
l'arête $E = [v_i,v_{i+1}]$, on distingue deux cas :

\begin{itemize}

\item Soit $v_j$ n'est pas incident à $E$ ($v_j \neq v_i$ et $v_j \neq
v_{i+1}$); dans ce cas $\lambda_j = 0$

\item Soit $v_j$ est incident à $E$ ($v_j = v_i$ ou $v_j = v_{i+1}$); dans ce
cas $\lambda_j$ est calculé comme étant le coefficient associé au sommet $v_j$
lors du calcul du barycentre du segment E, en considérant $p$ comme étant le
barycentre.

\end{itemize}