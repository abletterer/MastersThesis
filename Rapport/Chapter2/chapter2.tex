% \pagebreak[4]
% \hspace*{1cm}
% \pagebreak[4]
% \hspace*{1cm}
% \pagebreak[4]

\chapter{Mélange d'outils}

\graphicspath{ {Chapter2/Chapter2Figs/PNG/}
  {Chapter2/Chapter2Figs/PDF/} {Chapter2/Chapter2Figs/} }

Avant de commencer, précisons juste quelques notions par liées aux
déformations à base de cage:
\begin{itemize}
\item{\textit{Coordonnée :}} On définit la coordonnée d'un point comme
  étant une combinaison linéaire pondérée des positions sommets de sa
  cage.
\item{\textit{Poids :}} On appelle poids le coefficient associé à un
  sommet de la cage lors du calcul de la coordonnée d'un point de
  l'espace.\\
\end{itemize}

Chaque dimension d'outil permet de déformer l'espace de façon plus ou
moins locale. Les points et courbes sont plus adaptés aux déformations
locale, c'est-à-dire ne modifiant pas tout l'espace. Les surfaces et
volumes sont, quant à eux, utilisés lors de déformations plus
globales, modifiant tout l'espace contenu en leur intérieur.

Pourtant, sur le même ensemble de points, un utilisateur pourrait
souhaiter réaliser des déformations à la fois locales et
globales. Aussi, les techniques les plus utilisées ces dernières
années, à savoir les \textit{déformations à base de cage}, définissent
des déformations \textit{globales}. Une déformation est dite globale
lorsque la modification d'un des sommets de contrôle de l'outil influe
sur l'ensemble des points de l'espace, même de façon
infinitésimale. Il est nécessaire de calculer une coordonnée pour
chaque point de l'espace par rapport à tous les sommets de l'outil. De
plus, pour réaliser des déformations sur des zones précises, il faut
que l'outil soit composé d'un grand nombre de sommets, afin de
diminuer l'influence des sommets les plus éloignés de ces zones. On
peut donc dire qu'il existe un lien entre la précision d'une
déformation et le nombre de sommets de l'outil associé. Or plus il y a
de sommets composant l'outil, plus le temps de calcul des coordonnées
sera important. Il n'est donc pas possible de réaliser des
déformations sur des zones très précises, sans avoir à calculer de
coordonnées pour tous les points de l'espace.

C'est sur cette problématique que les idées de mélange d'outils ont
été introduites.

\section{Etat de l'art}

On peut citer \cite{JBPS11} comme étant le premier à avoir proposé une
méthode permettant de mélanger plusieurs outils de déformation de
différentes dimensions.  C'est sur celui-ci que nous avons commencé à
travailler car la méthode nous semblait proche de ce que nous
souhaitions réaliser. Une lecture plus approfondie de l'article nous a
fait nous rendre compte que le fonctionnement n'était pas celui que
nous souhaitions. En effet, s'il semble s'appuyer sur des outils ayant
des dimensions différentes en fonction des zones à déformer, la
gestion interne repose uniquement sur des déformations d'outils de
dimension 0 (points). L'aspect "multidimensionnel" est donc uniquement
présent pour imposer des contraintes supplémentaires sur les calculs
de coordonnées. Par exemple pour des sommets reliés par une arête
l'article définit que les poids (permettant le calcul des coordonnées)
évoluent de façon linéaire le long de cette arête. De plus, cette
méthode passe par une minimisation de l'énergie laplacienne,
nécessitant une discrétisation de l'espace. Or c'est quelque chose que
nous souhaiterions éviter, à cause du temps de calcul requis par ces
opérations.
\\

\cite{GPCP13} quant à lui, propose une méthode permettant le mélange
d'outil de même dimension, en s'intéressant particulièrement aux cas
des surfaces, à travers les déformations à base de cage.  Nous nous
sommes intéressés à cet article de par sa récente publication (2013),
sa proximité avec \cite{Hur12}, un travail réalisé par un étudiant en
Master ISI en 2012, et de l'utilisation de cages de déformation, le
modèle semblant le plus pertinent parmi les outils de dimension 2.
L'idée est de réaliser un assemblage de différentes cages collées
ensemble le long de leurs arêtes et de considérer les coordonnées d'un
point de l'espace non seulement par rapport à sa cage \textit{propre}
(à comprendre la cage englobant le point de l'espace) mais aussi par
rapport aux cages adjacentes à celle-ci.

La méthode de cet article propose au premier abord une formulation
très simple. Celui-ci définit que la position d'un point de l'espace
n'est plus simplement constituée d'une combinaison linéaire des
positions des sommets de sa cage propre, mais résulte d'une
interpolation linéaire entre les coordonnées calculées par rapport à
la cage propre et aux différentes cages \textit{jointure}. Une cage
jointure correspond à l'union des cages incidentes à un sommet de la
cage propre. L'avantage de ce genre de méthode est de localiser les
déformations en les limitant au voisinage de la cage incidente au
sommet déplacé. On peut donc avoir jusqu'à $n$ coordonnées différentes
pour un même point de l'espace, où $n$ correspond au nombre de sommets
de la cage propre. Ce qui au final fait perdre un des intérêts de la
méthode, au moins en partie, à savoir la réduction de la complexité en
temps de calcul.

Cependant cette formulation cache d'autres formulations qui semblent
résulter d'un procédé empirique, dont le cheminement n'est pas
expliqué dans l'article. Ce qui rend la compréhension de l'utilité de
ces fonctions assez difficile.

\section{Cheminement de l'article}
Dans un premier temps, nous avons voulu reproduire le cheminement de
\cite{GPCP13}, pour comprendre quelles étaient les motivations
derrière l'existence de chaque fonction. Nous sommes donc partis de
l'expression globale de l'article, qui exprime la position d'un point
de l'espace comme étant une pondération des coordonnées de la cage
propre et de la cage jointure :

\begin{equation}
  p = \beta~ T(p)  + (1 - \beta~) J(p) 
  \label{MELgen}
\end{equation}

Où $T(p)$ et $J(p)$ représentent les coordonnées du point p par
rapport à sa cage propre et à sa cage jointure respectivement, et
$\beta~$ représente la distance au bord de la cage propre. Pour
réduire la complexité du nombre de coordonnées qu'il sera nécessaire
de calculer, nous avons choisi de considérer une unique cage jointure
pour chaque point de l'espace, qui résulte de l'union de toutes les
cages incidentes aux sommets de la cage propre. Ainsi, pour chaque
point de l'espace le nombre de coordonnées à calculer sera de 2 au
maximum (une pour la cage propre, et une pour la cage jointure).

Le calcul de $\beta~$ était proposé par l'article. Il s'agit de juger
la distance d'un point de l'espace par rapport à chaque arête
incidente à sa cage propre et à une autre cage. Pour éviter de
calculer des distances euclidiennes, l'article propose de se baser sur
la coordonnée calculée pour chaque point $p$ par rapport aux sommets
de sa cage propre. Comme nous l'avons expliqué au début, la coordonnée
d'un point $p$ par rapport à une cage est calculée comme étant une
combinaison linéaire pondérée des positions des sommets de cette
cage. De ce fait, on peut considérer la distance de $p$ à un sommet de
la cage comme étant le poids qui a été associé à ce sommet lors du
calcul de la coordonée. Et par extension, on peut considérer la
distance de $p$ à une arête comme étant liée à la somme des poids
associés aux sommets incidents de cette arête. Plus précisément, plus
on est proche d'une arête, plus la somme des poids associés sommets
incidents à cette arête va être proche de 1. A l'inverse, plus on est
loin d'une arête, plus la somme des poids associés aux sommets de
cette arête va être proche de 0. La distance à une arête correspond
donc au complément à 1 de la somme des poids associés à chaque sommet
des arêtes incidentes à plusieurs cages :

\begin{equation}
  d_e~ = (1 - \sum_{v \in e} \lambda_v)
\end{equation}

Où $e$ correspond à une arête, $v$ un sommet de $e$ et $\lambda_v$ le
poids associé au sommet v. L'article considère $\beta~$ comme étant le
produit de la distance de $p$ à chaque arête de la cage propre qui
sont aussi incidentes à d'autres cages :

\begin{equation}
  \beta~ = f(\prod_{e \in C} d_e)
\end{equation}

Où $f(x)$ est une fonction de lissage (Equation \ref{MELlis}),
permettant de faire varier la taille de la zone d'infuence du mélange,
tout en conservant un estompement progressif :

\begin{equation}
  f(x) = \frac{1}{2} sin(\pi(\frac{x}{h}-\frac{1}{2})) + \frac{1}{2}
  \label{MELlis}
\end{equation}
Où $h \in~ ]0,1]$ représente la zone d'influence de l'arête, ce qui
permet de délimiter la zone où le mélange de coordonnées doit être
fait (figure \ref{MELpar}). Sur la fonction en elle-même, l'influence
de $h$ correspond à une contraction de la fonction du domaine [0,1] au
domaine [0,h] (Figure \ref{MELfon}).

\begin{figure}[h]
  \begin{center}
    \includegraphics[scale=0.35]{starCage-0-2}
    \includegraphics[scale=0.35]{starCage-0-4}
    \includegraphics[scale=0.35]{starCage-0-6}
    \caption{Fonction de bordure calculée pour 4 cages (arêtes
      noires). Les variations de couleur représentent les variations
      de valeur de $\beta~$ (la couleur bleu marine représentant une
      valeur de 0 et la couleur rouge une valeur de 1). En bleu les
      zones dites "de bordure" (i.e. points de l'espace proches d'une
      arête incidente à une autre cage). Valeurs de h : 0.2, 0.4 et
      0.6 pour les images à gauche, au centre et à droite
      respectivement.}
    \label{MELpar}
  \end{center}
\end{figure}

\begin{figure}[h]
  \begin{center}
    \begin{tikzpicture}
      \begin{axis}[colormap/bluered, width=0.37\textwidth,
        height=0.3\textwidth]
        \addplot gnuplot[scatter, samples=500, domain=0:0.2, mark=*,
        mark size=0.5]{1/2.*sin(pi*(x/0.2-1/2.))+1/2.};

        \addplot gnuplot[scatter, samples=500, domain=0.2:1, mark=*,
        mark size=0.5]{1};

        \draw ({axis cs:0.2,0}|-{rel axis cs:0,0}) -- ({axis
          cs:0.2,0}|-{rel axis cs:0,1}) node[near start,right] {$h =
          0.2$};
      \end{axis}
    \end{tikzpicture}
    \begin{tikzpicture}
      \begin{axis}[colormap/bluered, width=0.37\textwidth,
        height=0.3\textwidth]
        \addplot gnuplot[scatter, samples=500, domain=0:0.4, mark=*,
        mark size=0.5]{1/2.*sin(pi*(x/0.4-1/2.))+1/2.};

        \addplot gnuplot[scatter, samples=500, domain=0.4:1, mark=*,
        mark size=0.5]{1};

        \draw ({axis cs:0.4,0}|-{rel axis cs:0,0}) -- ({axis
          cs:0.4,0}|-{rel axis cs:0,1}) node[near start,right] {$h =
          0.4$};
      \end{axis}
    \end{tikzpicture}
    \begin{tikzpicture}
      \begin{axis}[colormap/bluered, width=0.37\textwidth,
        height=0.3\textwidth]
        \addplot gnuplot[scatter, samples=300, domain=0:0.6, mark=*,
        mark size=0.5]{1/2.*sin(pi*(x/0.6-1/2.))+1/2.};

        \addplot gnuplot[scatter, samples=300, domain=0.6:1, mark=*,
        mark size=0.5]{1};

        \draw ({axis cs:0.6,0}|-{rel axis cs:0,0}) -- ({axis
          cs:0.6,0}|-{rel axis cs:0,1}) node[near start,right] {$h =
          0.6$};
      \end{axis}
    \end{tikzpicture}
    \caption{Visualisation de la fonction f(x) pour différentes
      valeurs de h}
    \label{MELfon}
  \end{center}
\end{figure}

Comme $0 \leq \beta~ \leq 1$, on peut considérer $\beta~$ comme le
pourcentage d'utilisation des coordonnées calculées par rapport à la
cage propre. D'après l'équation \ref{MELgen}, lorsque $\beta~$ vaut 0
(i.e. $p$ se trouve sur une arête de la cage), la position de $p$
dépend uniquement de la coordonnée calculée par rapport à la cage
jointure $c_j$.

C'est là qu'apparaît le premier problème : Considérons une grille 2*2
constituée de 4 faces, chaque face constituant une cage. Toutes les
cages sont incidentes à un même sommet $s$. De ce fait, la zone autour
des arêtes incidentes à ce sommet sera fortement influencée par la
cage jointure composée de l'union des 4 cages. Afin que la cage
jointure reste un polygone, $s$ ne doit pas être considéré comme un
sommet de la cage jointure, car il fait partie de l'intérieur de
celle-ci. L'équation \ref{MELgen} nous indique que les zones les plus
proches des arêtes (i.e. là ou la valeur de $\beta~$ vaut 0), ne sont
influencées que par la cage jointure. Et que, par conséquent, $s$ n'a
aucune influence sur ces points. On peut voir sur la figure
\ref{MELjoi} que les points les plus proches des arêtes incidentes à
$s$ sont en bleu marine, ce qui signifie que $s$ n'a aucune influence
(ou une influence très faible) sur les points les plus proches de sa
position et une plus grande influence sur les points plus éloignés. Ce
qui est un comportement complètement contre-intuitif et la déformation
engendrée n'est pas du tout celle attendue.

\begin{figure}[h]
  \begin{center}
    \includegraphics[scale=0.35]{starCage-jointure}
    \includegraphics[scale=0.35]{starCage-jointure-deformation}
    \caption{ A gauche la grille avant déformation et à droite la même
      grille après déplacement de $s$. Les boules rouges représentent
      les sommets de la cage jointure et la boule bleue représente le
      sommet $s$. Les zones en bleu ne sont pas modifiées par la
      translation de $s$.}
    \label{MELjoi}
  \end{center}
\end{figure}

De là nous voyons la nécessité de rajouter un comportement spécifique
pour ces fameux sommets qui font partie de l'intérieur de leur cage
jointure. C'est une des fonctions proposées par l'article. Ils
résolvent ce problème au travers de l'expression d'une déformation à
base de points, associée au sommet $s$. A partir de là, nous avons
compris que la méthode proposée \cite{GPCP13} n'était que le résultat
d'une suite de résolutions de cas spécifiques. Ce qui ne nous
intéressait pas vraiment, car un de nos critères principaux était
l'utilisation d'une formulation simple permettant d'exprimer les
coordonnées de chaque point de façon claire. De plus, cette méthode
semble difficilement associable avec une méthode multidimensionnelle,
car elle se base sur une fusion d'outils (l'union des cages de
déformation), ce qui ne semble pas être une méthode directement
applicable pour des outils de différentes dimensions.

Nous pouvons néanmoins relever l'avancée qu'apporte cet article dans
le domaine des déformations à base de cages. En effet, grâce au
mélange de plusieurs cages, les déformations peuvent être locales,
tout en ayant un faible nombre de sommets composant les cages. De
plus, la possibilité d'utiliser conjointement plusieurs systèmes de
coordonnées différents permet de choisir le plus adapté aux
différentes déformations à effectuer.

C'est pour ça que la prochaine méthode s'inspire de cet article, en
ayant en tête cette idée de localisation de la déformation et de
limitation du temps de calcul des coordonnées.

\section{Méthode élaborée}
L'idée est de considérer des coordonnées \textit{étendues} pour chaque
cage, au lieu de considérer des unions de cages, et de réaliser un
mélange de coordonnées pour les points de l'espace qui sont sous
l'influence de plusieurs cages. On se base ici sur le fait que les
coordonnées MVC sont définies non seulement à l'intérieur, mais aussi
à l'extérieur du polygone de contrôle.

Pour vérifier la possibilité de réalisation d'une telle méthode, nous
avons commencé par travailler sur un exemple simple, une cage unique
déformant des points de l'espace à la fois à l'intérieur et à
l'extérieur grâce aux coordonnées MVC. Il s'agissait de voir le
comportement de la déformation, afin de savoir s'il était possible de
calculer des coordonnées à la fois à l'intérieur et à l'extérieur de
la cage, tout en permettant un passage lisse de l'un à l'autre. Car
dans la littérature, les coordonnées n'ont été utilisées que pour
déformer des points à l'intérieur du polygone et jusqu'à son bord.

Il se trouve que les coordonnées MVC sont $C^\infty$ partout, sauf au
niveau des sommets du polygone de contrôle, où elles ne sont que $C^0$
(Corr. 4.8, Prop. 3, \cite{HF06}). De plus, à l'extérieur de la cage,
on peut constater que les coordonnées ne s'estompent pas. Ceci est dû
à la partition de l'unité, qui est une propriété des coordonnées
barycentriques généralisées. Il est induit par la normalisation des
poids associés à chaque sommet (Equation \ref{MELnor}).

\begin{equation}
  \lambda_i(p) = \frac{w_i(p)}{\sum_{j=0}^n w_j(p)}
  \label{MELnor}
\end{equation}

Cette propriété définit que la somme des poids associés à chaque
sommet de la cage, pour un point $p$ donné, doit valoir 1 (Equation
\ref{MELsum}) :

\begin{equation}
  \sum_{i=0}^n \lambda_i(p) = 1
  \label{MELsum}
\end{equation}

Pour expliquer en quoi cela pose un problème, regardons la
construction des coordonnées MVC pour un point $p$ donné. Dans un
premier temps on évalue l'influence de chaque sommet $v_i$, dont le
calcul se fait par rapport au sommet voisins $v_{i-1}$ et $v_{i+1}$ :

\begin{equation}
  w_i(p) = \frac{tan(\alpha_{i-1}(p)/2) + tan(\alpha_{i}(p)/2)}{\|v_i - p\|}
\end{equation}



\begin{figure}[h]
  \begin{center}
    \scalebox{1} % Change this value to rescale the drawing.
    {
      \begin{pspicture}(0,-3.771875)(7.3034377,3.751875)
        \psdots[dotsize=0.2](2.1,3.631875)
        \psdots[dotsize=0.2](0.1,1.731875)
        \psdots[dotsize=0.2](1.68,0.411875)
        \psdots[dotsize=0.2](2.3,-3.128125)
        \psdots[dotsize=0.2](6.68,-1.748125)
        \psdots[dotsize=0.2](6.02,3.071875)
        \psdots[dotsize=0.2](3.78,1.391875)
        \psline[linewidth=0.04cm](2.32,-3.148125)(1.68,0.431875)
        \psline[linewidth=0.04cm](2.32,-3.128125)(6.72,-1.728125)
        \psline[linewidth=0.04cm](1.66,0.431875)(0.08,1.731875)
        \psline[linewidth=0.04cm](0.08,1.711875)(2.1,3.651875)
        \psline[linewidth=0.04cm](3.78,1.371875)(2.12,3.611875)
        \psline[linewidth=0.04cm](3.8,1.391875)(6.0,3.051875)
        \psline[linewidth=0.04cm](6.68,-1.768125)(6.04,3.071875)
        \psline[linewidth=0.04cm](2.1,3.631875)(5.98,3.091875)
        \psline[linewidth=0.04cm](3.78,1.371875)(6.68,-1.748125)
        \psline[linewidth=0.04cm](2.3,-3.168125)(3.78,1.371875)
        \psline[linewidth=0.04cm](1.7,0.391875)(3.76,1.351875)
        \psline[linewidth=0.04cm](0.1,1.711875)(3.82,1.391875)
        \rput{-135.0}(5.4728017,4.9906588){\psarc[linewidth=0.04](3.77,1.361875){0.5515433}{27.758541}{174.64418}}
        \usefont{T1}{ptm}{m}{n} \rput(2.2734375,-3.493125){\large
          $v_{i-1}$} \usefont{T1}{ptm}{m}{n}
        \rput(6.6734376,-2.213125){\large $v_i$}
        \usefont{T1}{ptm}{m}{n} \rput(5.9734373,3.486875){\large
          $v_{i+1}$} \usefont{T1}{ptm}{m}{n}
        \rput(3.9634376,0.366875){\large $\alpha_{i-1}$}
        \usefont{T1}{ptm}{m}{n} \rput(4.7234373,1.346875){\large
          $\alpha_i$} \usefont{T1}{ptm}{m}{n}
        \rput(3.8434374,1.926875){\large $p$}
      \end{pspicture}
    }
    \label{MELmvc}
    \caption{Visualisation du calcul de la coordonnée $w_i(p)$}
  \end{center}
\end{figure}

Où $w_i(p)$ représente le poids associé au sommet $v_i$ pour le point
$p$ et $\alpha_i(p)$ l'angle en $p$ du triangle $[p,v_i,v_{i+1}]$
(Figure \ref{MELmvc}). Les $w_i$ sont ensuite normalisés (Equation
\ref{MELnor}), afin d'obtenir des poids contenus dans le domaine
[0,1]. Le problème est lié à cette normalisation, car quand on éloigne
un point $p$ de la cage, si les $w_i(p)$ tendent vers 0 quand la
distance $\|v_i - p\|$ tend vers l'infini, leur somme aussi tend vers
0. De ce fait, les points extrémement éloignés de la cage seront aussi
fortement influencés par les déformations appliquées sur la cage. On
voit donc la nécessité de définir des zones d'influence autour de
chaque cage, afin de limiter leur champ d'action.

% ------------------------------------------------------------------------


%%% Local Variables: 
%%% mode: latex
%%% LaTeX-command: "latex -shell-escape"
%%% TeX-master: "../thesis"
%%% End: 
