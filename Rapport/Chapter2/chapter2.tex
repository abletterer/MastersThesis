%!TEX root = ../thesis.tex

% \pagebreak[4]
% \hspace*{1cm}
% \pagebreak[4]
% \hspace*{1cm}
% \pagebreak[4]

\chapter{Déformation à base de cage}

\graphicspath{ {Chapter2/Chapter2Figs/PNG/}
  {Chapter2/Chapter2Figs/PDF/} {Chapter2/Chapter2Figs/} }

Notre travail s'est porté particulièrement sur le mélange de différents outils
de dimension 2 (faces). La technique retenue pour cette dimension est celle des
déformations à base de cage. Ce choix a été motivé par l'efficacité de cette
méthode et de son utilisation dans le travail de \cite{GPCP13} (un travail
similaire à celui que nous souhaitions réaliser).

Afin de pouvoir rentrer dans les détails des méthodes existantes, ainsi que des
idées que nous avons eu, une petite introduction aux déformations à base de cage
est nécessaire. Ces déformations peuvent se faire dans $\mathbb{R}^2$ (la cage
est un polygone) et dans $\mathbb{R}^3$ (la cage est une surface), mais pour
simplifier la compréhension de la méthode, nous nous concentrerons sur des
déformations dans $\mathbb{R}^2$.

\section{Introduction} 

L'outil permettant de réaliser les déformations est un polygone. Ce polygone
peut-être de nature quelconque, sous réserve qu'il ne soit pas dégénéré (il doit
avoir au moins 3 sommets) et qu'il ne comporte pas d'auto-intersection. Par la
suite l'appelerons le \textit{polygone de contrôle}. L'idée est de définir une
position paramétrique des sommets d'un objet contenus à l'intérieur du polygone
de contrôle en fonction de la position de ses sommets. Cette méthode permet de
déformer un objet uniquement en manipulant les sommets du polygone de contrôle,
peu importe la complexité de cet objet. Cette déformation est dite globale, car
un sommet du polygone de contrôle a une influence sur tous les points de
l'espace (qui sont les sommets de l'objet) contenus à l'intérieur du polygone.
\\

Les premiers travaux travaillant là dessus ont été réalisés par \cite{JSW05} 
\cite{FKR05} (travaux réalisés simultanément), en utilisant la technique des
Mean Value Coordinates (MVC) (développée par \cite{Flo03}). Il s'agit de la
première méthode à avoir appliqué des coordonnées barycentriques généralisées
pour réaliser des déformations. Les coordonnées barycentriques généralisées
correspondent à la généralisation du calcul de coordonnées barycentriques, en ne
considérant plus des coordonnées calculées par rapport à un simplexe (triangle
dans $\mathbb{R}^2$), mais à une cellule (polygone dans $\mathbb{R}^2$).
Néanmoins cette méthode comporte un important défaut : les coordonnées calculées
peuvent être négatives (dans le cas d'un polygone concave par exemple), ce qui
peut amener à obtenir des résultats contre-intuitifs. \\

\cite{JMDGS07} ont déclaré que l'importance de ce problème devait amener à ne
pas utiliser cette technique pour l'animation de personnages. A la place, ils
ont proposé une approche différente, qui ne produisait pas de coordonnées
négatives : les Harmonic Coordinates (HC). Cette méthode a permis d'obtenir des
déformations plus localisées, mais au prix d'une complexité en temps de calcul
beaucoup plus importante et d'une discrétisation de l'espace assez fine. \\

Dans la même optique, \cite{LKCL07} proposent une amélioration des Mean Value
Coordinates qui les rend positives. L'idée derrière cela est de se baser sur un
test de visibilité pour supprimer les coordonnées négatives. Ils ont pu obtenir
des résultats similaires à \cite{JMDGS07} mais pour une complexité en temps de
calcul proche de celle des Mean Value Coordinates. \\

\cite{LLC08} ont constaté que les détails de la surface n'étaient pas préservés,
en particulier lors de déformations importantes. Ce qui les a amené à développer
une approche considérant plus de données lors de l'association entre la cage et
l'objet : les Green Coordinates (GC). Au lieu d'utiliser uniquement la position
des sommets de la cage, les GC utilisent aussi les normales aux faces de la
cage.

\section{Avantages}

\section{Limitations}