%!TEX root = ../thesis.tex

% \pagebreak[4]
% \hspace*{1cm}
% \pagebreak[4]
% \hspace*{1cm}
% \pagebreak[4]

\chapter{Déformation à base de cage}

\graphicspath{ {Chapter2/Chapter2Figs/PNG/}
  {Chapter2/Chapter2Figs/PDF/} {Chapter2/Chapter2Figs/} }

Le travail qui a été effectué s'est intéressé particulièrement au mélange de
différents outils de dimension 2 (faces). La technique retenue pour cette
dimension est celle des déformations à base de cage. Ce choix a été motivé par
l'efficacité de cette méthode et de son utilisation dans le travail de
\cite{GPCP13} (un travail similaire à celui que nous avons réalisé).

\section{Fonctionnement} 

\section{Méthodes de calcul des coordonnées} 

\section{Avantages}

\section{Limitations}