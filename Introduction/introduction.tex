%%% Thesis Introduction --------------------------------------------------
\chapter{Introduction}
\ifpdf
    \graphicspath{{Introduction/IntroductionFigs/PNG/}{Introduction/IntroductionFigs/PDF/}{Introduction/IntroductionFigs/}}
\else
    \graphicspath{{Introduction/IntroductionFigs/EPS/}{Introduction/IntroductionFigs/}}
\fi

La modélisation géométrique a permis, dans un premier temps, de représenter des modèles virtuels. 
Mais le besoin d'éditer ces modèles a amené la modélisation géométrique a évoluer et à instaurer des outils permettant de modifier ces modèles. 
Ainsi sont nés les outils de "déformation". Dans le cadre de notre travail, nous ne nous sommes que penchés sur les déformations dites "spatiales"

La déformation spatiale consiste à déformer un objet en modifiant son espace ambiant. 
On notera \cite{Bar84} et \cite{SP86} comme étant les premiers à avoir introduit ce type de déformation. 
Ce procédé a un avantage considérable, la modification de l'espace. 
En effet comme la déformation est réalisée sur chaque point de l'espace de façon indépendante, elle n'est pas liée à la représentation interne de l'objet.

La première partie de ce travail a consisté en la réalisation d'un mélange de différents outils de déformation. 
En parallèle, une étude a été faite sur les différents outils de déformation, basé sur \cite{GB08}, pour déterminer le meilleur outil associé à chaque dimension (point, courbe, surface, volume).

%%% ----------------------------------------------------------------------


%%% Local Variables: 
%%% mode: latex
%%% TeX-master: "../thesis"
%%% End: 
