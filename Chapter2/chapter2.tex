% \pagebreak[4]
% \hspace*{1cm}
% \pagebreak[4]
% \hspace*{1cm}
% \pagebreak[4]

\chapter{Mélange d'outils}

\graphicspath{{Chapter2/Chapter2Figs/PNG/}{Chapter2/Chapter2Figs/PDF/}{Chapter2/Chapter2Figs/}}

La déformation à base de cage est une déformation dite globale,
c'est-à-dire que le déplacement d'un des sommets de contrôle de la
cage influe sur l'ensemble des points de l'espace, même de façon
infinitésimale. De ce fait, il est nécessaire de calculer des
coordonnées pour chaque point de l'espace par rapport à l'ensemble des
sommets de la cage. La précision des déplacements est liée au nombre
de sommets de la cage, en effet, plus la cage contient de sommets,
plus les points de l'espace vont pouvoir être déplacés de façon
précise. On voit donc directement apparaitre un problème important, la
précision d'une déformation est liée au nombre de sommets de la cage,

\section{Etat de l'art}
On peut citer \cite{JBPS11} comme étant le premier a avoir proposé une
méthode permettant de mélanger différents outils de déformation de
différentes dimensions.  C'est sur celui-ci que nous avons commencé à
travailler car la méthode nous semblait proche de ce que nous
souhaitions réaliser. Une lecture plus approfondie de l'article nous a
fait nous rendre compte que le fonctionnement n'était pas celui que
nous souhaitions.  En effet, s'il semble s'appuyer sur des outils
ayant des dimensions différentes en fonction des zones à déformer, la
gestion interne repose uniquement sur des déformations d'outils de
dimension 0 (points).  L'aspect "multidimensionnel" est donc
uniquement présent pour imposer des contraintes sur les calculs de
coordonnées. De plus, les calculs réalisés lors de l'assocation des
points de l'espace aux sommets des outils nécessitent une
discrétisation de l'espace, ce qui est une méthode que l'on souhaitait
éviter, du fait de la complexité importante en temps de calcul.

\cite{GPCP13} quant à lui, propose une méthode permettant le mélange
d'outil de dimension 2 (surfaces). Nous nous sommes intéressés à cet
article de par sa récente publication (2013), sa proximité avec
\cite{Hur12}, un travail réalisé par un étudiant en Master ISI en
2012, et de l'utilisation de cages de déformation, le modèle semblant
le plus pertinent parmi les outils de dimension 2.
% ------------------------------------------------------------------------


%%% Local Variables: 
%%% mode: latex
%%% TeX-master: "../thesis"
%%% End: 
