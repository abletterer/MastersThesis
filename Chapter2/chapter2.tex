% \pagebreak[4]
% \hspace*{1cm}
% \pagebreak[4]
% \hspace*{1cm}
% \pagebreak[4]

\chapter{Mélange d'outils}

\graphicspath{ {Chapter2/Chapter2Figs/PNG/}
  {Chapter2/Chapter2Figs/PDF/} {Chapter2/Chapter2Figs/} }

Chaque dimension d'outil permet de déformer l'espace d'une autre
manière. Les points et courbes sont plus adaptés aux déformations
réalisées par rapport à un axe (rotation, torsion, fuselage). Les
surfaces et volumes sont, quant à eux, utilisés lors de déformations
plus générales.

Pourtant, sur le même ensemble de points, un utilisateur pourrait
souhaiter réaliser des déformations par rapport à un axe, mais aussi
plus générales. Aussi, les techniques les plus utilisées ces dernières
années, les \textit{cages de déformation} par exemple, définissent des
déformations \textit{globales}. Une déformation est dite globale
lorsque la modification d'un des sommets de contrôle de l'outil influe
sur l'ensemble des points de l'espace, même de façon
infinitésimale. Il est donc nécessaire de calculer des coordonnées
pour chaque point de l'espace par rapport à tous les sommets de
l'outil. De plus, pour réaliser des déformations sur des zones
précises, il faut que l'outil soit composé d'un grand nombre de
sommets, afin de diminuer l'influence des sommets les plus éloignés de
ces zones. On peut donc dire qu'il existe un lien entre la précision
d'une déformation et le nombre de sommets de l'outil associé. Or plus
il y a de sommets composant l'outil, plus le temps de calcul des
coordonnées sera important. Il n'est donc pas possible de réaliser des
déformations sur des zones très précises, sans avoir à calculer de
coordonnées pour tous les points de l'espace.

C'est sur cette problématique que les idées de mélange d'outils ont
été introduites.

\section{Etat de l'art}

On peut citer \cite{JBPS11} comme étant le premier a avoir proposé une
méthode permettant de mélanger plusieurs outils de déformation de
différentes dimensions.  C'est sur celui-ci que nous avons commencé à
travailler car la méthode nous semblait proche de ce que nous
souhaitions réaliser. Une lecture plus approfondie de l'article nous a
fait nous rendre compte que le fonctionnement n'était pas celui que
nous souhaitions. En effet, s'il semble s'appuyer sur des outils ayant
des dimensions différentes en fonction des zones à déformer, la
gestion interne repose uniquement sur des déformations d'outils de
dimension 0 (points). L'aspect "multidimensionnel" est donc uniquement
présent pour imposer des contraintes supplémentaires sur les calculs
de coordonnées. Par exemple pour des sommets reliés par une arête
l'article définit que les poids (permettant le calcul des coordonnées)
évoluent de façon linéaire le long de cette arête. De plus, cette
méthode passe par une minimisation de l'énergie laplacienne,
nécessitant une discrétisation de l'espace. Or c'est quelque chose que
nous souhaiterions éviter, à cause du temps de calcul requis par ces
opérations.

\cite{GPCP13} quant à lui, propose une méthode permettant le mélange
d'outil de même dimension, en s'intéressant particulièrement aux cas
des surfaces, à travers les déformations à base de cage.  Nous nous
sommes intéressés à cet article de par sa récente publication (2013),
sa proximité avec \cite{Hur12}, un travail réalisé par un étudiant en
Master ISI en 2012, et de l'utilisation de cages de déformation, le
modèle semblant le plus pertinent parmi les outils de dimension 2.
L'idée est de réaliser un assemblage de différentes cages collées
ensembles le long de leurs arêtes et de considérer les coordonnées
d'un point de l'espace non seulement par rapport à sa cage
\textit{propre} (à comprendre la cage englobant le point de l'espace)
mais aussi par rapport aux cages adjacentes à celle-ci.  La méthode de
cet article propose au premier abord une formulation très claire. La
position d'un point de l'espace n'est plus simplement constituée d'une
combinaison linéaire des positions des sommets de sa cage propre, mais
un mélange entre les coordonnées calculées par rapport à la cage
propre et par rapport aux différentes cages \textit{jointure}.
L'article définit une cage jointure comme l'ensemble des cages
incidentes à un sommet de la cage propre.  L'avantage de ce genre de
méthode est de localiser les déformations en les limitant au voisinage
de la cage incidente au sommet déplacé.  On peut donc avoir jusqu'à
$n$ coordonnées différentes pour un même point de l'espace, où $n$
correspond au nombre de sommets de la cage propre. Ce qui au final
fait un peu perdre un des intérêts de la méthode, à savoir la
réduction de la complexité en temps de calcul.
% ------------------------------------------------------------------------


%%% Local Variables: 
%%% mode: latex
%%% TeX-master: "../thesis"
%%% End: 
