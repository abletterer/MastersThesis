% \pagebreak[4]
% \hspace*{1cm}
% \pagebreak[4]
% \hspace*{1cm}
% \pagebreak[4]

\chapter{Mélange d'outils}

\graphicspath{ {Chapter2/Chapter2Figs/PNG/}
  {Chapter2/Chapter2Figs/PDF/} {Chapter2/Chapter2Figs/} }

Chaque dimension d'outil permet de déformer l'espace d'une autre
manière. Les points et courbes sont plus adaptés aux déformations par
rapport à un axe (rotation, torsion, fuselage). Les surfaces et
volumes sont, quant à eux, utilisés lors de déformations plus
générales. Pourtant, sur le même ensensemble de points, un utilisateur
pourrait souhaiter réaliser des opérations relevant à la fois des
points et courbes, ainsi que 

\section{Etat de l'art}

On peut citer \cite{JBPS11} comme étant le premier a avoir proposé une
méthode permettant de mélanger différents outils de déformation de
différentes dimensions.  C'est sur celui-ci que nous avons commencé à
travailler car la méthode nous semblait proche de ce que nous
souhaitions réaliser. Une lecture plus approfondie de l'article nous a
fait nous rendre compte que le fonctionnement n'était pas celui que
nous souhaitions.  En effet, s'il semble s'appuyer sur des outils
ayant des dimensions différentes en fonction des zones à déformer, la
gestion interne repose uniquement sur des déformations d'outils de
dimension 0 (points).  L'aspect "multidimensionnel" est donc
uniquement présent pour imposer des contraintes sur les calculs de
coordonnées.


\cite{GPCP13} quant à lui, propose une méthode permettant le mélange
d'outil de même dimension, en s'intéressant particulièrement aux cas
des surfaces, à travers les déformations à base de cage. Nous nous
sommes intéressés à cet article de par sa récente publication (2013),
sa proximité avec \cite{Hur12}, un travail réalisé par un étudiant en
Master ISI en 2012, et de l'utilisation de cages de déformation, le
modèle semblant le plus pertinent parmi les outils de dimension
2. L'idée est de réaliser un assemblage de différentes cages collées
ensembles le long de leurs arêtes et de considérer les coordonnées
d'un point de l'espace non seulement par rapport à sa cage
\textit{propre} (à comprendre la cage englobant le point de l'espace)
mais aussi par rapport aux cages adjacentes à celle-ci.
% ------------------------------------------------------------------------


%%% Local Variables: 
%%% mode: latex
%%% TeX-master: "../thesis"
%%% End: 
